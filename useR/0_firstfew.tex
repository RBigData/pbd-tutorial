%%%%%%%%%%%%%%%%%%%%%%%%%%%%%%%%%%%%%%%%
%%     Title and ToC
%%%%%%%%%%%%%%%%%%%%%%%%%%%%%%%%%%%%%%%%
% titlepage
\frame{
  \maketitle
}

\begin{frame}[noframenumbering]
\frametitle{Affiliations and Support}
{\small
The pbdR Core Team\\ \url{http://r-pbd.org}
\\[.4cm]
Wei-Chen Chen\footnote{\tiny{Computer Science and Mathematics Division, Oak Ridge National Laboratory, Oak Ridge, TN}}, 
George Ostrouchov$^{1,2}$, 
Pragneshkumar Patel\footnote{\tiny{Remote Data Analysis and Visualization Center, University of Tennessee, Knoxville, TN}}, 
Drew Schmidt$^1$
\\[.4cm]
Ostrouchov, Patel, and Schmidt were supported in part by the project
``NICS Remote Data Analysis and Visualization Center''
funded by the Office of Cyberinfrastructure of the
U.S. National Science Foundation
under Award No. ARRA-NSF-OCI-0906324 for NICS-RDAV center.\\[.4cm]
Chen and Ostrouchov were supported in part by the project
``Visual Data Exploration and Analysis of Ultra-large Climate Data''
funded by U.S. DOE Office of Science
under Contract No. DE-AC05-00OR22725.\\
}
\end{frame}

\begin{frame}
\frametitle{About This Presentation}
 \begin{block}{Downloads}
  This presentation and supplemental materials are available at:
  \begin{center}
  \url{http://r-pbd.org/handouts}
  \end{center}
 \end{block}
\end{frame}


\begin{frame}
\frametitle{About This Presentation}
 \begin{block}{\emph{Speaking Serial R with a Parallel Accent}}
  The content of this presentation is based in part on the \pkg{pbdDEMO} 
vignette \emph{Speaking Serial R with a Parallel Accent}\\[.4cm]
  \url{https://github.com/wrathematics/pbdDEMO/blob/master/inst/doc/pbdDEMO-guide.pdf?raw=true}\\[.4cm]
  It contains more examples, and sometimes added detail.
 \end{block}
\end{frame}


\begin{frame}
\frametitle{About This Presentation}
 \begin{block}{Installation Instructions}
  Installation instructions for setting up a pbdR environment are available:
  \begin{center}
  \url{http://r-pbd.org/install.html}
  \end{center}
  This includes instructions for installing R, MPI, and pbdR.
 \end{block}
\end{frame}



\begin{frame}%[allowframebreaks=0.8]
\frametitle{About This Presentation}
 \begin{block}{Conventions}
%   \begin{itemize}
%     \item 
    We use:
    \begin{itemize}
    \item ``{\Huge$ .$}'' as a decimal mark
    \item ``{\Huge$,$}'' as order of magnitude separator
    \end{itemize}
    \begin{center}
    \begin{tabular}{|l|l|l|}\hline
      Example & Yes & No \\\hline
      One million & 1,000,000 & 1.000.000\\
      One half & 0.5 & 0,5\\
      One thousand and one half & $1,000.5$ & $1.000,5$\\\hline
    \end{tabular}
    \end{center}
%     \item We will use special suffixes to denote distributed objects (ones not stored entirely on a single processor).\\
%     \code{.spmd} denotes a distributed object, while\\
%     \code{.dmat} denotes a distributed object which is of class \code{ddmatrix}\\
%     No suffix means the object is global (common to all processors)\\[.2cm]
%     Neither of these suffices carries semantic meaning.
%     \end{itemize}
 \end{block}
\end{frame}



% \begin{frame}[fragile]
% \frametitle{About This Presentation}
%  \begin{block}{Conventions For Code Presentation}
% We will use two different forms of syntax highlighting.  One for displaying results from an interactive R session:
% \begin{lstlisting}[backgroundcolor=\color{white},basicstyle=\ttfamily\color{dkgray}\scriptsize,keywordstyle=\color{black}, 
%   commentstyle=\color{orange},stringstyle=\color{mauve}]
% R> "interactive"
% [1] "interactive"
% \end{lstlisting}
% and one for presenting R scripts
% \begin{lstlisting}
% "not interactive"
% \end{lstlisting}
%  \end{block}
% \end{frame}



\begin{frame}[noframenumbering,shrink]
\frametitle{Contents}
\small
\tableofcontents[hideallsubsections]
\end{frame}

\setcounter{framenumber}{0}
