\section[pbdDMAT Eg's]{pbdDMAT Examples}

\hidenum
\begin{frame}[noframenumbering]
\frametitle{Contents}
 \tableofcontents[currentsection,hideothersubsections,sectionstyle=show/hide]
\end{frame}
\shownum

\subsection{Compression with Principal Components Analysis}

\setcounter{excount}{0}

\begin{frame}
  \begin{block}{Example \countex :  PCA}\pause
    Compute the principal components of a distributed matrix.  Retain only a subset of the rotated data, the greatest number of columns which will retain no more than 90\% of the variation of the original dataset.
  \end{block}
\end{frame}

\begin{frame}[fragile]
  \begin{block}{Example \showex :  PCA SPMD Algorithm}\pause
    \begin{enumerate}
     \item Set good random seed and generate $10,000\times 250$ \code{ddmatrix}
     \item Compute PCA rotation with scaling using \code{prcomp()}.
     \item Determine the first $i$ columns which retain no more than 90\% of the original variation.
     \item Retain only the first $i$ columns of the rotated data.
    \end{enumerate}
  \end{block}
\end{frame}


\begin{frame}[fragile]
  \begin{exampleblock}{Example \showex :  PCA Code}\pause
\begin{lstlisting}
n <- 1e4
p <- 250

comm.set.seed(diff=T)
dx <- Hnorm(dim=c(n, p), bldim=c(4,4), mean=100, sd=25)

pca <- prcomp(x=dx, retx=TRUE, scale=TRUE)
prop_var <- cumsum(pca$sdev)/sum(pca$sdev)
i <- max(min(which(prop_var > 0.9)) - 1, 1)

new_dx <- pca$x[, 1:i]
\end{lstlisting}
  \end{exampleblock}
\end{frame}


\begin{frame}[fragile,shrink]
  \begin{exampleblock}{Example \showex :  PCA Batch Execution}\pause
  Locate the \pkg{pbdDEMO} example script \code{pca.r} and execute:
\vspace{-.4cm}
  \begin{lstlisting}[language=Sh]
### At the shell prompt, run the demo with 4 processors
### Use Rscript.exe for Windows systems
mpirun -np 4 Rscript pca.r
\end{lstlisting}
Sample output:\vspace{-.4cm}
\begin{lstlisting}[language=Sh]
DENSE DISTRIBUTED MATRIX
---------------------------
@Data:			-9.81e-01, -8.39e-01, -1.33e-01,  6.33e-02, ...
Process grid:		2x2
Global dimension:	10000x221
(max) Local dimension:	5000x112
Blocking:		4x4
BLACS CTXT:		0


Number of columns retained:	 221 
Percentage of columns retained: 0.884 
\end{lstlisting}
  \end{exampleblock}
\end{frame}





\subsection{Predictions with Linear Regression}

\begin{frame}
  \begin{block}{Example \countex :  Regression}\pause
    Fit the linear model $\by = \bX \bbeta + \bepsilon$ and make a prediction on new $x$ data using this model.
  \end{block}
\end{frame}

\begin{frame}[fragile]
  \begin{block}{Example \showex :  Regression SPMD Algorithm}\pause
    \begin{enumerate}
     \item Set good random seed and generate $1250\times 40$ \code{ddmatrix} $x$ and $1250\times 1$ \code{ddmatrix} $y$
     \item Fit the linear model using \code{lm.fit()}.
     \item Generate new $x$ data.
     \item Compute the estimated $\hat{y} = x_{\text{new}}*\beta$.
    \end{enumerate}
  \end{block}
\end{frame}


\begin{frame}[fragile]
  \begin{exampleblock}{Example \showex :  Regression Code}\pause
\begin{lstlisting}
comm.set.seed(1234, diff=TRUE)
dx <- Hnorm(c(n, p), bldim=bldim, mean=mean, sd=sd)
dy <- Hunif(c(n, 1), bldim=bldim, min=ymin, max=ymax)

mdl <- lm.fit(dx, dy)

dx.new <- Hnorm(c(1, p), bldim=bldim, mean=mean, sd=sd)

pred <- dx.new %*% mdl$coefficients
\end{lstlisting}
  \end{exampleblock}
\end{frame}


\begin{frame}[fragile]
  \begin{exampleblock}{Example \showex :  Regression Batch Execution}\pause
  Locate the \pkg{pbdDEMO} example script \code{ols_dmat.r} and execute:
\vspace{-.4cm}
  \begin{lstlisting}[language=Sh]
### At the shell prompt, run the demo with 4 processors
### Use Rscript.exe for Windows systems
mpirun -np 4 Rscript ols_dmat.r
\end{lstlisting}
Sample output:\vspace{-.4cm}
\begin{lstlisting}[language=Sh]
The predicted y value is: 84.7432227923963 
\end{lstlisting}
  \end{exampleblock}
\end{frame}