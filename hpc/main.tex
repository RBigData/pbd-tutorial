%%% For the presentation slides
% \documentclass{beamer}

%%% Handouts
\documentclass[handout,style=authortitle]{beamer}
% \usepackage{handoutWithNotes}
% \usepackage{pgfpages}
% \pgfpagesuselayout{3 on 1 with notes big}[a4paper,border shrink=7mm]
% \pgfpageslogicalpageoptions{1}{border code=\pgfusepath{stroke}}
% \pgfpageslogicalpageoptions{2}{border code=\pgfusepath{stroke}}
% \pgfpageslogicalpageoptions{3}{border code=\pgfusepath{stroke}}



\include{preamble}
\usepackage{caption}
\usepackage{subcaption}


% \title[Introduction to pbdR]{NICS Spring Training:\\  Introduction to R and pbdR}
\title[Introduction to pbdR]{Introducing R: \\  from Your Laptop to HPC and Big Data}
% \subtitle{NICS Spring Training}
\author[\color{white}{http://r-pbd.org} \hspace{2.4cm} pbdR Core Team]{Drew Schmidt\\ Remote Data Analysis and Visualization Center\\ University of Tennessee, Knoxville\vspace{-.8cm}}
\date{June 17, 2013 \\[.5cm] \centering\includegraphics[scale=.6]{pics/logos}} 
\logo{\begin{tabular}{r}\includegraphics[height=.34cm]{pics/utk_logo.png} \\ \includegraphics[height=.34cm]{pics/ornl.jpg}\end{tabular}}

\newcommand{\mytitlea}{Introduction to pbdR \hspace{3cm} \insertframenumber\,/\,\inserttotalframenumber}
\newcommand{\mytitleb}{Introduction to pbdR}

%%%%%%%%%%%%%%%%%%%%%%%%%%%%%%%%
\begin{document}

%%%%%%%%%%%%%%%%%%%%%%%%%%%%%%%%%%%%%%%%
%%     Title and ToC
%%%%%%%%%%%%%%%%%%%%%%%%%%%%%%%%%%%%%%%%
% titlepage
\frame{
  \maketitle
}

\begin{frame}[noframenumbering]
\frametitle{Affiliations and Support}
{\small
The pbdR Core Team\\ \url{http://r-pbd.org}
\\[.4cm]
Wei-Chen Chen\footnote{\tiny{Computer Science and Mathematics Division, Oak Ridge National Laboratory, Oak Ridge, TN}}, 
George Ostrouchov$^{1,2}$, 
Pragneshkumar Patel\footnote{\tiny{Remote Data Analysis and Visualization Center, University of Tennessee, Knoxville, TN}}, 
Drew Schmidt$^1$
\\[.4cm]
Ostrouchov, Patel, and Schmidt were supported in part by the project
``NICS Remote Data Analysis and Visualization Center''
funded by the Office of Cyberinfrastructure of the
U.S. National Science Foundation
under Award No. ARRA-NSF-OCI-0906324 for NICS-RDAV center.\\[.4cm]
Chen and Ostrouchov were supported in part by the project
``Visual Data Exploration and Analysis of Ultra-large Climate Data''
funded by U.S. DOE Office of Science
under Contract No. DE-AC05-00OR22725.\\
}
\end{frame}

\begin{frame}
\frametitle{About This Presentation}
 \begin{block}{Downloads}
  This presentation and supplemental materials are available at:
  \begin{center}
  \url{http://r-pbd.org/tutorial}
  \end{center}
  Sample R scripts and pbs job scripts available on Nautilus from:\\
  \centering\code{/lustre/medusa/mschmid3/tutorial/scripts.tar.gz}
 \end{block}
\end{frame}


\begin{frame}
\frametitle{About This Presentation}
 \begin{block}{\emph{Speaking Serial R with a Parallel Accent}}
  The content of this presentation is based in part on the \pkg{pbdDEMO} 
vignette \emph{Speaking Serial R with a Parallel Accent}\\[.4cm]
  \url{http://goo.gl/HZkRt}\\[.4cm]
  It contains more examples, and sometimes added detail.
 \end{block}
\end{frame}


\begin{frame}
\frametitle{About This Presentation}
 \begin{block}{Installation Instructions}
  Installation instructions for setting up a pbdR environment are available:
  \begin{center}
  \url{http://r-pbd.org/install.html}
  \end{center}
  This includes instructions for installing R, MPI, and pbdR.
 \end{block}
\end{frame}



\begin{frame}[noframenumbering]
\frametitle{Contents}
\small
\tableofcontents[hideallsubsections]
\end{frame}

\setcounter{framenumber}{0}

\section{Introduction to R}

\hidenum
\begin{frame}[noframenumbering]
\frametitle{Contents}
 \tableofcontents[currentsection,hideothersubsections,sectionstyle=show/hide]
\end{frame}
\shownum



\subsection{What is R?}

\begin{frame}
  \begin{block}{What is R?}\pause
  \begin{itemize}[<+-|alert@+>]
    \item \emph{lingua franca} for data analytics and statistical computing.
    \item Part programming language, part data analysis package.
    \item Dialect of S (Bell Labs).
    \item Syntax designed for data.
%     \item Functional programming paradigms, lazy evaluation, and lexical 
scoping semantics, and 2 official OOP systems.
  \end{itemize}
\end{block}
\end{frame}

\begin{frame}
  \begin{block}{Who uses R?}\pause
   Google, Pfizer, Merck, Bank of America, 
Shell\footnote{\url{
https://www.nytimes.com/2009/01/07/technology/business-computing/07program.html?
_r=0}}, 
   
Oracle\footnote{\url{
http://www.oracle.com/us/corporate/features/features-oracle-r-enterprise-498732.
html}},
   Facebook, bing, Mozilla, 
okcupid\footnote{\url{
http://www.revolutionanalytics.com/what-is-open-source-r/companies-using-r.php}}
,
   
ebay\footnote{\url{
http://blog.revolutionanalytics.com/2012/09/using-r-in-production-industry-exper
ts-share-their-experiences.html}},
   
kickstarter\footnote{\url{
http://blog.revolutionanalytics.com/2012/09/kickstarter-facilitates-50m-in-indie
-game-funding.html}},
   the New York 
Times\footnote{\url{
http://blog.revolutionanalytics.com/2012/05/nyt-charts-the-facebook-ipo-with-r.h
tml}}
  \end{block}
\end{frame}

\begin{frame}
  \begin{block}{Language Paradigms}\pause
  \begin{center}
    \includegraphics[scale=.35]{../common/pics/languages}
  \end{center}
  \end{block}
\end{frame}

\begin{frame}
  \begin{block}{Data Types}\pause
  \begin{itemize}[<+-|alert@+>]
    \item Storage:  logical, int, double, double complex, character
    \item Structures:  vector, matrix, array, list, dataframe
    \item Caveats:  (Logical) \code{TRUE}, \code{FALSE}, \code{NA}
  \end{itemize}
  For the remainder of the tutorial, we will restrict ourselves to real number 
matrix computations.
\end{block}
\end{frame}




\subsection{Syntax for Data Science}


\begin{frame}[fragile]
\begin{block}{High Level Syntax}\pause
\begin{lstlisting}
x <- matrix(rnorm(30), nrow=10)
x <- x[-1, 2:5]
x <- log(abs(x) + 1)
xtx <- t(x) %*% x
ans <- svd(solve(xtx))
\end{lstlisting}
\end{block}
\end{frame}


\begin{frame}
  \begin{block}{More than just a Matlab clone\dots}\pause
  \begin{itemize}[<+-|alert@+>]
    \item Data science (machine learning, statistics, data mining, \dots) is 
mostly matrix algebra.  \\[.2cm]
     So what about Matlab/Python/Julia/\dots ?
    \item Depends on your ``religion'' 
    \item As a \emph{data analysis} package, R is king.
  \end{itemize}
\end{block}
\end{frame}



\begin{frame}[fragile]
\begin{block}{High Level Syntax \emph{for Data}}\pause
\begin{lstlisting}
pca <- prcomp(x, retx=TRUE, scale=TRUE)
prop_var <- cumsum(pca$sdev)/sum(pca$sdev)
i <- min(which(prop_var > 0.9)) - 1

y <- pca$x[, 1:i]
\end{lstlisting}
\end{block}
\end{frame}

















\section{Parallel Hardware and R}

\hidenum
\begin{frame}[noframenumbering]
\frametitle{Contents}
 \tableofcontents[currentsection,hideothersubsections,sectionstyle=show/hide]
\end{frame}
\shownum

\subsection{Parallel Hardware}

% \begin{frame}
% \begin{block}{A}
%   
% \end{block}
% \end{frame}


\begin{frame}
\begin{block}{Three Basic Flavors of Hardware}
    \includegraphics[width=0.95\textwidth]{../common/pics/ParallelHardware1.pdf}
\end{block}
\end{frame}

\begin{frame}
\begin{block}{Your Laptop or Desktop}
    \includegraphics[width=0.95\textwidth]{../common/pics/ParallelHardware2.pdf}
\end{block}
\end{frame}

\begin{frame}
\begin{block}{A Server or Cluster}
    \includegraphics[width=0.95\textwidth]{../common/pics/ParallelHardware3.pdf}
\end{block}
\end{frame}

\begin{frame}
\begin{block}{Server to Supercomputer}
    \includegraphics[width=0.95\textwidth]{../common/pics/ParallelHardware4.pdf}
\end{block}
\end{frame}

\begin{frame}
\begin{block}{Knowing the Right Words}
    \includegraphics[width=0.95\textwidth]{../common/pics/ParallelHardware5.pdf}
\end{block}
\end{frame}

\begin{frame}
\begin{block}{``Native'' Programming Models and Tools}
    \includegraphics[width=0.95\textwidth]{../common/pics/ParallelHardware6.pdf}
\end{block}
\end{frame}

\subsection{R Interfaces to Parallel Hardware}

\begin{frame}
\begin{block}{R Interfaces to Native Tools}
    \includegraphics[width=0.95\textwidth]{../common/pics/ParallelHardware7.pdf}
\end{block}
\end{frame}

\begin{frame}
\begin{block}{30+ Years of Parallel Computing Research}
    \includegraphics[width=0.95\textwidth]{../common/pics/ParallelHardware8.pdf}
\end{block}
\end{frame}

\begin{frame}
\begin{block}{Last 10 years of Advances}
    \includegraphics[width=0.95\textwidth]{../common/pics/ParallelHardware9.pdf}
\end{block}
\end{frame}

\begin{frame}
\begin{block}{Putting It All Together Challenge}
    
\includegraphics[width=0.95\textwidth]{../common/pics/ParallelHardware10.pdf}
\end{block}
\end{frame}

\begin{frame}
\begin{block}{pbdR Focus on Data Parallelism}
    
\includegraphics[width=0.95\textwidth]{../common/pics/ParallelHardware11.pdf}
\end{block}
\end{frame}

\include{2_example}
\section{pbdR}

\hidenum
\begin{frame}[noframenumbering]
\frametitle{Contents}
 \tableofcontents[currentsection,hideothersubsections,sectionstyle=show/hide]
\end{frame}
\shownum

\subsection{Problems with R}

\begin{frame}
%  \addtocounter{framenumber}{-1}
  \begin{block}{Problems with R}\pause
  We \emph{love} R!  However\dots
  \begin{itemize}[<+-|alert@+>]
    \item Slow.
    \item If you don't know what you're doing, it's \emph{really} slow.
    \item Performance improvements usually for small machines.
    \item Very ram intensive.
    \item Chokes on big data.
  \end{itemize}
  \end{block}
\end{frame}

\begin{frame}
%  \addtocounter{framenumber}{-1}
  \begin{block}{Problems with R: Big Data}\pause
  One of R's biggest problems is an indexing limitation:
  \begin{itemize}[<+-|alert@+>]
    \item Any one R object must (at present) be indexed by a 32-bit integer.
    \item Largest vector/matrix:  16gb
    \item Largest square matrix:  $46340\times 46340$
  \end{itemize}
  \end{block}
\end{frame}

\begin{frame}[shrink]
  \begin{block}{R and Parallelism}
    The solution to many of R's problems is parallelism.  However \dots\vspace{-.4cm}
   \begin{center}
    \begin{minipage}[t]{.95\textwidth}
    \begin{block}{\centering What we have}
      \begin{enumerate}[<+-|alert@+>]
	\item Mostly serial.
	\item Parallelism mostly not distributed.
	\item Data parallelism mostly explicit.
      \end{enumerate}
    \end{block}
    \end{minipage}
    \\\pause
    \begin{minipage}[t]{.95\textwidth}
    \begin{block}{\centering What we want}
      \begin{enumerate}[<+-|alert@+>]
        \item Mostly parallel.
        \item Mostly distributed.
        \item Mostly implicit.
      \end{enumerate}
    \end{block}
    \end{minipage}
    \end{center}
    \end{block}
\end{frame}


\subsection{The pbdR Project}

\begin{frame}[squeeze]
  \begin{block}{Programming with Big Data in R (pbdR)}\pause
  Goals:  \emph{Productivity, Portability, Performance}\\[.4cm]\pause
  Our Approach:
  \begin{itemize}[<+-|alert@+>]
    \item Series of \emph{free}\footnote{MPL, BSD, and GPL licensed} R packages.
    \item Scalable, big data analytics with high-level syntax.
    \item Implicit management of distributed data details.
    \item Methods have syntax \emph{identical} to R.
    \item Powered by state of the art numerical libraries (MPI, ScaLAPACK, PBLAS, BLACS, LAPACK, BLAS, \dots)
  \end{itemize}
  \end{block}
\end{frame}

\begin{frame}[shrink]
  \begin{block}{pbdR Packages}
    \begin{center}
	\includegraphics[width=7cm, height=7cm]{pics/pbdpacks}
    \end{center}
  \end{block}
\end{frame}


\begin{frame}[shrink]
  \begin{block}{pbdR Packages --- http://code.r-pbd.org}\pause
  Released to CRAN:
  \begin{itemize}[<+-|alert@+>]
    \item \pkg{pbdMPI}: MPI bindings (explicit, low-level)
    \item \pkg{pbdSLAP}: Foreign library (just install it, nothing to use)
    \item \pkg{pbdBASE}: Compiled code (used by DMAT, also for devs)
    \item \pkg{pbdDMAT}: Distributed matrices (mostly implicit, high-level)
    \item \pkg{pbdNCDF4}: Parallel NetCDF4 reader
    \item \pkg{pbdDEMO}: Package demonstrations, examples, vignette written in textbook style
  \end{itemize}
%     \\[.2cm]
    Future Development:
  \begin{itemize}[<+-|alert@+>]
    \item \pkg{pbdADIOS}: Wrappers for ADIOS middleware
    \item Profiling tools
    \item Client/server interface for interactive sessions
    \item Something for you\dots?
  \end{itemize}
  \end{block}
\end{frame}

\begin{frame}
  \begin{block}{SPMD}\pause
  The pbdR Packages enable high-level ``Single Program/Multiple Data'' (SPMD) programming:
    \begin{itemize}
      \item SPMD is a programming \emph{paradigm}.
      \item Arguably the simplest extension of serial programming.
      \item Sort of like trying to explain breathing \dots
      \item Not to be confused with SIMD.
      \item SPMD utilizes MIMD architecture computers.
      \item Only one program is written, executed in batch independently on all processors.
      \item Different processors are autonomous; there is no manager.
%       \item Like ``Map/Reduce'', you probably used it without knowing it even had a name.
    \end{itemize}
  \end{block}
\end{frame}


\begin{frame}[fragile]
  \begin{block}{SPMD}\pause
      SPMD codes are run in batch (non-interactively):
\begin{lstlisting}[backgroundcolor=\color{white},keywordstyle=\color{black},title=From the Shell]
mpirun -np 4 Rscript my_script.R
\end{lstlisting}
  \end{block}
\end{frame}


\begin{frame}[fragile]
  \begin{block}{Example Syntax}\pause
  \begin{lstlisting}
x <- x[-1, 2:5]
xtx <- t(x) %*% x
ans <- svd(solve(xtx))
  \end{lstlisting}
  \begin{center}
  \pause Look familiar?\\[.4cm] \pause
  \emph{The above runs on 1 core with R or 10,000 cores with pbdR}
  \end{center}
  \end{block}
\end{frame}



\subsection{Installing pbdR}

\begin{frame}[fragile]
  \begin{block}{Installation}\pause
  Installing pbdR is about as easy as possible, and generally amounts to:
  \begin{lstlisting}
install.packages(pbdMPI)
install.packages(pbdNCDF4)
install.packages(pbdSLAP)
install.packages(pbdBASE)
install.packages(pbdDMAT)
install.packages(pbdDEMO)
  \end{lstlisting}
  But this assumes you have MPI installed on your system\dots
  \end{block}
\end{frame}


\begin{frame}[fragile]
  \begin{block}{NICS Allocation}\pause
  Instead, consider getting an allocation on Nautilus:
  \begin{center}
  \url{http://www.nics.tennessee.edu/getting-an-allocation}\\
  \includegraphics[scale=.3]{pics/nics}
  \end{center}
  \end{block}
\end{frame}
\section[Break]{Brief Intermission}
\hidenum
\begin{frame}[noframenumbering]
\frametitle{Brief Intermission}
  \begin{block}{Brief Intermission}
  \begin{center}
     {\Large Questions?  Comments?}\\[.4cm]
     Don't forget to talk to us at our discussion group: \url{http://group.r-pbd.org/}\\[.4cm]
     Don't have an allocation with us?  \\
     \url{http://www.nics.tennessee.edu/getting-an-allocation}
  \end{center}
  \end{block}
\end{frame}
\shownum


\section[pbdMPI]{Introduction to pbdMPI}

\hidenum
\begin{frame}[noframenumbering]
\frametitle{Contents}
 \tableofcontents[currentsection,hideothersubsections,sectionstyle=show/hide]
\end{frame}
\shownum

\subsection{Basic MPI}

\begin{frame}
  \begin{block}{Message Passing Interface (MPI)}\pause
    \begin{itemize}
      \item \textit{MPI}: Standard for managing communications (data and instructions) between different nodes/computers.
      \item \textit{Implementations}:  OpenMPI, MPICH2, Cray MPT, \dots
      \item Enables parallelism on distributed machines.
      \item \textit{Communicator}: manages communications between processors.
    \end{itemize}
  \end{block}
\end{frame}


\begin{frame}
  \begin{block}{Common MPI Operations (1 of 2)}\pause
    \begin{itemize}
      \item \textbf{Managing a Communicator}:  Create and destroy communicators.\\
      \code{init()} --- initialize communicator\\
      \code{finalize()} --- shut down communicator(s)
      \\[.4cm]
      \item \textbf{Rank query}: determine the processor's position in the communicator.\\
      \code{comm.rank()} --- ``who am I?''\\
      \code{comm.size()} --- ``how many of us are there?''\\[.4cm]
      \item \textbf{Barrier}: ``computation wall''; no processor can proceed until \emph{all} processors can proceed.\\
      \code{barrier()}\\[.4cm]
    \end{itemize}
  \end{block}
\end{frame}


\begin{frame}[fragile]
  \begin{exampleblock}{Quick Example 1}
   \begin{center}
\begin{lstlisting}[title=Rank Query]
library(pbdMPI, quiet = TRUE)
init()

myRank <- comm.rank()
comm.print(myRank, all.rank=TRUE)

finalize()
\end{lstlisting}

\begin{lstlisting}[title=Sample Output]
COMM.RANK = 0
[1] 0
COMM.RANK = 1
[1] 1
\end{lstlisting}

    \end{center}
  \end{exampleblock}
\end{frame}

\begin{frame}[shrink]
  \begin{block}{Common MPI Operations (2 of 2)}\pause
    \begin{itemize}
      \item \textbf{Reduction}:  each processor has a number \code{x.spmd}; add all of them up, find the largest/smallest, \dots .\\
      \code{reduce(x.spmd, op='sum')} --- reduce to one\\
      \code{allreduce(x.spmd, op='sum')} --- reduce to all\\[.4cm]
      \item \textbf{Gather}: each processor has a number; create a new object on some processor containing all of those numbers.\\
      \code{gather(x.spmd)} --- gather to one\\
      \code{allgather(x.spmd)} --- gather to all\\[.4cm]
      \item \textbf{Broadcast}: one processor has a number \code{x.spmd} that every other processor should also have.\\
      \code{bcast(x.spmd)}
      \\[.4cm]
    \end{itemize}
  \end{block}
\end{frame}

\begin{frame}[fragile,shrink]
  \begin{exampleblock}{Quick Example 2}
   \begin{center}
\begin{lstlisting}
library(pbdMPI, quiet = TRUE)
init()

comm.set.seed(diff=TRUE)

n <- sample(1:10, size=1)

sm <- allreduce(n, op='sum')
comm.print(sm)

gt <- allgather(n)
comm.print(unlist(gt))

finalize()
\end{lstlisting}

\begin{lstlisting}[title=Sample Output]
COMM.RANK = 0
[1] 10
COMM.RANK = 0
[1] 2 8
\end{lstlisting}
    \end{center}
  \end{exampleblock}
\end{frame}





\subsection{The SPMD Data Structure}

\begin{frame}[fragile,shrink]
  \begin{block}{The \code{SPMD} Data Structure}\pause
  Throughout the examples, we will make use of the \code{SPMD} distributed matrix structure.
  \begin{enumerate}
    \item \code{SPMD} is \emph{distributed}.  No one processor owns all of the matrix.
%     \item \code{SPMD} is \emph{balanced}.  Every processor owns (roughly) the same amount of data.
    \item \code{SPMD} is \emph{non-overlapping}. Any row owned by one processor is owned by no other processors.
    \item \code{SPMD} is \emph{row-contiguous}.  If a processor owns one element of a row, it owns the entire row.
    \item \code{SPMD} is globally \emph{row-major}, locally \emph{column-major}.
    \item The last row of the local storage of a processor is adjacent (by global row) to the first row of the local storage of next processor (by communicator number) that owns data.
    \item \code{SPMD} is (relatively) easy to understand, but can lead to bottlenecks if you have many more columns than rows.
  \end{enumerate}
  \end{block}
\end{frame}

\begin{frame}
\begin{exampleblock}{Understanding SPMD:  Global Matrix}
\begin{align*}
x &= \left[
      \begin{array}{lllllllll}
      x_{11} & x_{12} & x_{13} & x_{14} & x_{15} & x_{16} & x_{17} & x	_{18} & x_{19}\\
      x_{21} & x_{22} & x_{23} & x_{24} & x_{25} & x_{26} & x_{27} & x	_{28} & x_{29}\\
      x_{31} & x_{32} & x_{33} & x_{34} & x_{35} & x_{36} & x_{37} & x	_{38} & x_{39}\\
      x_{41} & x_{42} & x_{43} & x_{44} & x_{45} & x_{46} & x_{47} & x	_{48} & x_{49}\\
      x_{51} & x_{52} & x_{53} & x_{54} & x_{55} & x_{56} & x_{57} & x	_{58} & x_{59}\\
      x_{61} & x_{62} & x_{63} & x_{64} & x_{65} & x_{66} & x_{67} & x	_{68} & x_{69}\\
      x_{71} & x_{72} & x_{73} & x_{74} & x_{75} & x_{76} & x_{77} & x	_{78} & x_{79}\\
      x_{81} & x_{82} & x_{83} & x_{84} & x_{85} & x_{86} & x_{87} & x	_{88} & x_{89}\\
      x_{91} & x_{92} & x_{93} & x_{94} & x_{95} & x_{96} & x_{97} & x	_{98} & x_{99}
      \end{array}
\right]_{9\times 9}
\end{align*}
\begin{align*}
\text{Processors = }
      \begin{array}{llllll}
      \color{g11}0 & \color{g12}1 & \color{g13}2 & \color{g21}3 & \color{g22}4 & \color{g23}5
      \end{array}
\end{align*}
\end{exampleblock}
\end{frame}


\begin{frame}
\begin{exampleblock}{Understanding SPMD:  Load Balanced SPMD}
\begin{align*}
x &= \left[
      \begin{array}{lllllllll}
      \color{g11}x_{11} & \color{g11}x_{12} & \color{g11}x_{13} & \color{g11}x_{14} & \color{g11}x_{15} & \color{g11}x_{16} & \color{g11}x_{17} & \color{g11}x_{18} & \color{g11}x_{19}\\
      %
      \color{g11}x_{21} & \color{g11}x_{22} & \color{g11}x_{23} & \color{g11}x_{24} & \color{g11}x_{25} & \color{g11}x_{26} & \color{g11}x_{27} & \color{g11}x_{28} & \color{g11}x_{29}\\\hline
      %
      \color{g12}x_{31} & \color{g12}x_{32} & \color{g12}x_{33} & \color{g12}x_{34} & \color{g12}x_{35} & \color{g12}x_{36} & \color{g12}x_{37} & \color{g12}x_{38} & \color{g12}x_{39}\\
      %
      \color{g12}x_{41} & \color{g12}x_{42} & \color{g12}x_{43} & \color{g12}x_{44} & \color{g12}x_{45} & \color{g12}x_{46} & \color{g12}x_{47} & \color{g12}x_{48} & \color{g12}x_{49}\\\hline
      %
      \color{g13}x_{51} & \color{g13}x_{52} & \color{g13}x_{53} & \color{g13}x_{54} & \color{g13}x_{55} & \color{g13}x_{56} & \color{g13}x_{57} & \color{g13}x_{58} & \color{g13}x_{59}\\
      %
      \color{g13}x_{61} & \color{g13}x_{62} & \color{g13}x_{63} & \color{g13}x_{64} & \color{g13}x_{65} & \color{g13}x_{66} & \color{g13}x_{67} & \color{g13}x_{68} & \color{g13}x_{69}\\\hline
      %
      \color{g21}x_{71} & \color{g21}x_{72} & \color{g21}x_{73} & \color{g21}x_{74} & \color{g21}x_{75} & \color{g21}x_{76} & \color{g21}x_{77} & \color{g21}x_{78} & \color{g21}x_{79}\\\hline
      %
      \color{g22}x_{81} & \color{g22}x_{82} & \color{g22}x_{83} & \color{g22}x_{84} & \color{g22}x_{85} & \color{g22}x_{86} & \color{g22}x_{87} & \color{g22}x_{88} & \color{g22}x_{89}\\\hline
      %
      \color{g23}x_{91} & \color{g23}x_{92} & \color{g23}x_{93} & \color{g23}x_{94} & \color{g23}x_{95} & \color{g23}x_{96} & \color{g23}x_{97} & \color{g23}x_{98} & \color{g23}x_{99}\\
      \end{array}
\right]_{9\times 9}
\end{align*}
\begin{align*}
\text{Processors = }
      \begin{array}{llllll}
      \color{g11}0 & \color{g12}1 & \color{g13}2 & \color{g21}3 & \color{g22}4 & \color{g23}5
      \end{array}
\end{align*}
\end{exampleblock}
\end{frame}

\begin{frame}[shrink]
\begin{exampleblock}{Understanding SPMD:  Local View}
\begin{align*}
\left[\begin{array}{lllllllll}
      \color{g11}x_{11} & \color{g11}x_{12} & \color{g11}x_{13} & \color{g11}x_{14} & \color{g11}x_{15} & \color{g11}x_{16} & \color{g11}x_{17} & \color{g11}x_{18} & \color{g11}x_{19}\\
      \color{g11}x_{21} & \color{g11}x_{22} & \color{g11}x_{23} & \color{g11}x_{24} & \color{g11}x_{25} & \color{g11}x_{26} & \color{g11}x_{27} & \color{g11}x_{28} & \color{g11}x_{29}
\end{array}\right]_{2\times 9}
\\
\left[\begin{array}{lllllllll}
      \color{g12}x_{31} & \color{g12}x_{32} & \color{g12}x_{33} & \color{g12}x_{34} & \color{g12}x_{35} & \color{g12}x_{36} & \color{g12}x_{37} & \color{g12}x_{38} & \color{g12}x_{39}\\
      \color{g12}x_{41} & \color{g12}x_{42} & \color{g12}x_{43} & \color{g12}x_{44} & \color{g12}x_{45} & \color{g12}x_{46} & \color{g12}x_{47} & \color{g12}x_{48} & \color{g12}x_{49}
\end{array}\right]_{2\times 9}
\\
\left[\begin{array}{lllllllll}
      \color{g13}x_{51} & \color{g13}x_{52} & \color{g13}x_{53} & \color{g13}x_{54} & \color{g13}x_{55} & \color{g13}x_{56} & \color{g13}x_{57} & \color{g13}x_{58} & \color{g13}x_{59}\\
      \color{g13}x_{61} & \color{g13}x_{62} & \color{g13}x_{63} & \color{g13}x_{64} & \color{g13}x_{65} & \color{g13}x_{66} & \color{g13}x_{67} & \color{g13}x_{68} & \color{g13}x_{69}
\end{array}\right]_{2\times 9}
\\
\left[\begin{array}{lllllllll}
      \color{g21}x_{71} & \color{g21}x_{72} & \color{g21}x_{73} & \color{g21}x_{74} & \color{g21}x_{75} & \color{g21}x_{76} & \color{g21}x_{77} & \color{g21}x_{78} & \color{g21}x_{79}
\end{array}\right]_{1\times 9}
\\
\left[\begin{array}{lllllllll}
      \color{g22}x_{81} & \color{g22}x_{82} & \color{g22}x_{83} & \color{g22}x_{84} & \color{g22}x_{85} & \color{g22}x_{86} & \color{g22}x_{87} & \color{g22}x_{88} & \color{g22}x_{89}
\end{array}\right]_{1\times 9}
\\
\left[\begin{array}{lllllllll}
      \color{g23}x_{91} & \color{g23}x_{92} & \color{g23}x_{93} & \color{g23}x_{94} & \color{g23}x_{95} & \color{g23}x_{96} & \color{g23}x_{97} & \color{g23}x_{98} & \color{g23}x_{99}\\
\end{array}\right]_{1\times 9}
\end{align*}
\begin{align*}
\text{Processors = }
      \begin{array}{llllll}
      \color{g11}0 & \color{g12}1 & \color{g13}2 & \color{g21}3 & \color{g22}4 & \color{g23}5
      \end{array}
\end{align*}
\end{exampleblock}
\end{frame}

\begin{frame}
\begin{exampleblock}{Understanding SPMD:  Non-Balanced SPMD}
\begin{align*}
x &= \left[
      \begin{array}{lllllllll}
      \\\hline
      \color{g12}x_{11} & \color{g12}x_{12} & \color{g12}x_{13} & \color{g12}x_{14} & \color{g12}x_{15} & \color{g12}x_{16} & \color{g12}x_{17} & \color{g12}x_{18} & \color{g12}x_{19}\\
      %
      \color{g12}x_{21} & \color{g12}x_{22} & \color{g12}x_{23} & \color{g12}x_{24} & \color{g12}x_{25} & \color{g12}x_{26} & \color{g12}x_{27} & \color{g12}x_{28} & \color{g12}x_{29}\\
      %
      \color{g12}x_{31} & \color{g12}x_{32} & \color{g12}x_{33} & \color{g12}x_{34} & \color{g12}x_{35} & \color{g12}x_{36} & \color{g12}x_{37} & \color{g12}x_{38} & \color{g12}x_{39}\\
      %
      \color{g12}x_{41} & \color{g12}x_{42} & \color{g12}x_{43} & \color{g12}x_{44} & \color{g12}x_{45} & \color{g12}x_{46} & \color{g12}x_{47} & \color{g12}x_{48} & \color{g12}x_{49}\\\hline
      %%%%
      \color{g13}x_{51} & \color{g13}x_{52} & \color{g13}x_{53} & \color{g13}x_{54} & \color{g13}x_{55} & \color{g13}x_{56} & \color{g13}x_{57} & \color{g13}x_{58} & \color{g13}x_{59}\\
      %
      \color{g13}x_{61} & \color{g13}x_{62} & \color{g13}x_{63} & \color{g13}x_{64} & \color{g13}x_{65} & \color{g13}x_{66} & \color{g13}x_{67} & \color{g13}x_{68} & \color{g13}x_{69}\\\hline
      %%%%
      \color{g21}x_{71} & \color{g21}x_{72} & \color{g21}x_{73} & \color{g21}x_{74} & \color{g21}x_{75} & \color{g21}x_{76} & \color{g21}x_{77} & \color{g21}x_{78} & \color{g21}x_{79}\\\hline\hline
      %%%%
      \color{g23}x_{81} & \color{g23}x_{82} & \color{g23}x_{83} & \color{g23}x_{84} & \color{g23}x_{85} & \color{g23}x_{86} & \color{g23}x_{87} & \color{g23}x_{88} & \color{g23}x_{89}\\
      %
      \color{g23}x_{91} & \color{g23}x_{92} & \color{g23}x_{93} & \color{g23}x_{94} & \color{g23}x_{95} & \color{g23}x_{96} & \color{g23}x_{97} & \color{g23}x_{98} & \color{g23}x_{99}\\
      \end{array}
\right]_{9\times 9}
\end{align*}
\begin{align*}
\text{Processors = }
      \begin{array}{llllll}
      \color{g11}0 & \color{g12}1 & \color{g13}2 & \color{g21}3 & \color{g22}4 & \color{g23}5
      \end{array}
\end{align*}
\end{exampleblock}
\end{frame}

\begin{frame}[shrink]
\begin{exampleblock}{Understanding SPMD:  Local View}
\begin{align*}
\left[\begin{array}{lllllllll}
      &&&&&&&&\hspace{4.55cm} 
\end{array}\right]_{0\times 9}
\\
\left[\begin{array}{lllllllll}
      \color{g12}x_{11} & \color{g12}x_{12} & \color{g12}x_{13} & \color{g12}x_{14} & \color{g12}x_{15} & \color{g12}x_{16} & \color{g12}x_{17} & \color{g12}x_{18} & \color{g12}x_{19}\\
      %
      \color{g12}x_{21} & \color{g12}x_{22} & \color{g12}x_{23} & \color{g12}x_{24} & \color{g12}x_{25} & \color{g12}x_{26} & \color{g12}x_{27} & \color{g12}x_{28} & \color{g12}x_{29}\\
      %
      \color{g12}x_{31} & \color{g12}x_{32} & \color{g12}x_{33} & \color{g12}x_{34} & \color{g12}x_{35} & \color{g12}x_{36} & \color{g12}x_{37} & \color{g12}x_{38} & \color{g12}x_{39}\\
      %
      \color{g12}x_{41} & \color{g12}x_{42} & \color{g12}x_{43} & \color{g12}x_{44} & \color{g12}x_{45} & \color{g12}x_{46} & \color{g12}x_{47} & \color{g12}x_{48} & \color{g12}x_{49}\\
\end{array}\right]_{4\times 9}
\\
\left[\begin{array}{lllllllll}
      \color{g13}x_{51} & \color{g13}x_{52} & \color{g13}x_{53} & \color{g13}x_{54} & \color{g13}x_{55} & \color{g13}x_{56} & \color{g13}x_{57} & \color{g13}x_{58} & \color{g13}x_{59}\\
      %
      \color{g13}x_{61} & \color{g13}x_{62} & \color{g13}x_{63} & \color{g13}x_{64} & \color{g13}x_{65} & \color{g13}x_{66} & \color{g13}x_{67} & \color{g13}x_{68} & \color{g13}x_{69}\\
\end{array}\right]_{2\times 9}
\\
\left[\begin{array}{lllllllll}
      \color{g21}x_{71} & \color{g21}x_{72} & \color{g21}x_{73} & \color{g21}x_{74} & \color{g21}x_{75} & \color{g21}x_{76} & \color{g21}x_{77} & \color{g21}x_{78} & \color{g21}x_{79}
\end{array}\right]_{1\times 9}
\\
\left[\begin{array}{lllllllll}
    &&&&&&&&\hspace{4.55cm} 
\end{array}\right]_{2\times 9}
\\
\left[\begin{array}{lllllllll}
      \color{g23}x_{81} & \color{g23}x_{82} & \color{g23}x_{83} & \color{g23}x_{84} & \color{g23}x_{85} & \color{g23}x_{86} & \color{g23}x_{87} & \color{g23}x_{88} & \color{g23}x_{89}\\
      \color{g23}x_{91} & \color{g23}x_{92} & \color{g23}x_{93} & \color{g23}x_{94} & \color{g23}x_{95} & \color{g23}x_{96} & \color{g23}x_{97} & \color{g23}x_{98} & \color{g23}x_{99}\\
\end{array}\right]_{2\times 9}
\end{align*}
\begin{align*}
\text{Processors = }
      \begin{array}{llllll}
      \color{g11}0 & \color{g12}1 & \color{g13}2 & \color{g21}3 & \color{g22}4 & \color{g23}5
      \end{array}
\end{align*}
\end{exampleblock}
\end{frame}



\begin{frame}[fragile,shrink]
  \begin{block}{Quick Comments for Using pbdMPI}\pause
    \begin{enumerate}
      \item Start by loading the package:
\vspace{-.4cm}
\begin{lstlisting}
library(pbdMPI, quiet = TRUE)
\end{lstlisting}
      \item Always initialize before starting and finalize when finished:
\vspace{-.4cm}
\begin{lstlisting}
init()
# ...
finalize()
\end{lstlisting}
      \item Use \code{comm.set.seed(diff=TRUE)} to generate independent streams by L'Ecuyer's method.  Use \code{comm.set.seed(diff=FALSE)} to set a common seed among all processors.
      \item Local pieces of \code{SPMD} distributed objects will be given the suffix \code{.spmd} to visually help distinguish them from global objects.  This suffix carries no semantic meaning.
    \end{enumerate}
  \end{block}
\end{frame}












\section[pbdMPI eg's]{Examples Using pbdMPI}

\hidenum
\begin{frame}[noframenumbering]
\frametitle{Contents}
 \tableofcontents[currentsection,hideothersubsections,sectionstyle=show/hide]
\end{frame}
\shownum


\subsection{pbdMPI Example: Monte Carlo Simulation}

\begin{frame}[shrink]
  \begin{block}{Example \countex :  Monte Carlo Simulation}\pause
  Sample $N$ uniform observations $(x_i, y_i)$ in the unit square $[0, 1]\times [0,1]$.  Then
\begin{equation*}
\pi \approx 4\left(\frac{\#\ Inside\ Circle}{\#\ Total}\right) = 4\left(\frac{\text{\color{blue}{\# Blue}}}{\text{\color{blue}{\# Blue}}+\text{\color{red}{\# Red}}}\right)
\label{eqn:pi}
\end{equation*}
  \begin{center}
   \includegraphics[scale=.25]{pics/pi} 
  \end{center}
  \end{block}
\end{frame}


\begin{frame}[fragile]
  \begin{block}{Example \showex :  Monte Carlo Simulation SPMD Algorithm}\pause
    \begin{enumerate}
     \item Let $n$ be big-ish; we'll take $n=50,000$.
     \item Generate an $n\times 2$ matrix $x$ of standard uniform observations.
     \item Count the number of rows satisfying $x^2 + y^2 \leq 1$
     \item Ask everyone else what their answer is; sum it all up.
     \item Take this new answer, multiply by 4 and divide by $n$
     \item If my rank is 0, print the result.
    \end{enumerate}
  \end{block}
\end{frame}


\begin{frame}[fragile,scale,shrink]
  \begin{exampleblock}{Example \showex :  Monte Carlo Simulation Code}\pause
\begin{lstlisting}[title=Serial Code]
N <- 50000
X <- matrix(runif(N * 2), ncol=2)
r <- sum(rowSums(X^2) <= 1)
PI <- 4*r/N
print(PI)
\end{lstlisting}

\begin{lstlisting}[title=Parallel Code]
library(pbdMPI, quiet = TRUE)
init()
comm.set.seed(diff=TRUE)

N.spmd <- 50000 / comm.size()
X.spmd <- matrix(runif(N.spmd * 2), ncol = 2)
r.spmd <- sum(rowSums(X.spmd^2) <= 1)
r <- allreduce(r.spmd)
PI <- 4*r/(N.spmd * comm.size())
comm.print(PI)

finalize()
\end{lstlisting}
  \end{exampleblock}
\end{frame}

\begin{frame}[fragile]
  \begin{block}{Note}\pause
    For the remainder, we will exclude loading, init, and finalize calls.
  \end{block}
\end{frame}










\subsection{pbdMPI Example: Sample Covariance}

\begin{frame}
  \begin{block}{Example \countex :  Sample Covariance}\pause
  \begin{align*}
    cov(x_{n\times p}) = \frac{1}{n-1}\sum_{i=1}^n\left(x_i-\mu_x\right)\left(x_i-\mu_x\right)^T
  \end{align*}
  \end{block}
\end{frame}


\begin{frame}
  \begin{block}{Example \showex :  Sample Covariance SPMD Algorithm}\pause
    \begin{enumerate}
     \item Determine the total number of rows $N$.
     \item Compute the vector of column means of the full matrix.
     \item Subtract each column's mean from that column's entries in each local matrix.
     \item Compute the crossproduct locally and reduce.
     \item Divide by $N-1$.
    \end{enumerate}
  \end{block}
\end{frame}


\begin{frame}[fragile,shrink]
  \begin{exampleblock}{Example \showex :  Sample Covariance Code}\pause
\begin{lstlisting}[title=Serial Code]
N <- nrow(X)
mu <- colSums(X) / N

X <- sweep(X, STATS=mu, MARGIN=2)
Cov.X <- crossprod(X.spmd) / (N-1)

print(Cov.X)
\end{lstlisting}
  
\begin{lstlisting}[title=Parallel Code]
N <- allreduce(nrow(X.spmd), op="sum")
mu <- allreduce(colSums(X.spmd) / N, op="sum")

X.spmd <- sweep(X.spmd, STATS=mu, MARGIN=2)
Cov.X <- allreduce(crossprod(X.spmd), op="sum") / (N-1)

comm.print(Cov.X)
\end{lstlisting}
  \end{exampleblock}
\end{frame}







\subsection{pbdMPI Example: Linear Regression}

\begin{frame}
  \begin{block}{Example \countex :  Linear Regression}\pause
      Find $\bbeta$ such that
      \begin{align*}
      \by = \bX\bbeta + \bepsilon
      \end{align*}

      When $\bX$ is full rank,
      \begin{align*}
      \hat{\bbeta} = (\bX^T\bX)^{-1}\bX^T\by \label{math:ols}
      \end{align*}
  \end{block}
\end{frame}


\begin{frame}
  \begin{block}{Example \showex :  Linear Regression SPMD Algorithm}\pause
    \begin{enumerate}
     \item Locally, compute $tx = x^T$
     \item Locally, compute $A = tx * x$. Query every other processor for this result and sum up all the results.
     \item Locally, compute $B = tx * y$.  Query every other processor for this result and sum up all the results.
     \item Locally, compute $A^{-1} * B$
    \end{enumerate}
  \end{block}
\end{frame}


\begin{frame}[fragile,shrink]
  \begin{exampleblock}{Example \showex :  Linear Regression Code}\pause
\begin{lstlisting}[title=Serial Code]
tX <- t(X)
A <- tX %*% X
B <- tX %*% y

ols <- solve(A) %*% B
\end{lstlisting}
  
\begin{lstlisting}[title=Parallel Code]
tX.spmd <- t(X.spmd)
A <- allreduce(tX.spmd %*% X.spmd, op = "sum")
B <- allreduce(tX.spmd %*% y.spmd, op = "sum")

ols <- solve(A) %*% B
\end{lstlisting}
  \end{exampleblock}
\end{frame}
\include{5_dmat}
% % % % \section[Iris]{In-Depth Example Examining the Iris Dataset with pbdR}
% % % % 
% % % % \hidenum
% % % % \begin{frame}[noframenumbering]
% % % % \frametitle{Contents}
% % % %  \tableofcontents[currentsection,hideothersubsections,sectionstyle=show/hide]
% % % % \end{frame}
% % % % \shownum
% % % % 
% % % % \subsection{Examining the Iris Dataset}
% % % % 
% % % % \begin{frame}[fragile]
% % % %   \begin{block}{The Iris Dataset}\pause
% % % % \begin{lstlisting}
% % % % rm(list = ls())                   # Clean environment
% % % % 
% % % % library(pbdMPI, quiet = TRUE)     # Load library
% % % % if(comm.size() != 4)
% % % %   comm.stop("4 processors are required.")
% % % % 
% % % % ### Load data
% % % % X <- as.matrix(iris[, -5])        # Dimension 150 by 4
% % % % X.cid <- as.numeric(iris[, 5])    # True id
% % % % 
% % % % ### Distribute data
% % % % jid <- get.jid(nrow(X))
% % % % X.spmd <- X[jid,]                 # SPMD row-major format
% % % % \end{lstlisting}
% % % % \end{block}
% % % % \end{frame}
% % % % 
% % % % 
% % % % \begin{frame}[fragile]
% % % %   \begin{block}{Standardizing}\pause
% % % % \begin{lstlisting}
% % % % ### Standardized
% % % % N <- allreduce(nrow(X.spmd))             # 150
% % % % p <- ncol(X.spmd)                        # 4
% % % % mu <- allreduce(colSums(X.spmd / N))
% % % % X.std <- sweep(X.spmd, 2, mu, FUN = "-") # Substract mean
% % % % std <- sqrt(allreduce(colSums(X.std^2 / (N - 1))))
% % % % X.std <- sweep(X.std, 2, std, FUN = "/") # Divide standard error
% % % % \end{lstlisting}
% % % % \end{block}
% % % % \end{frame}
% % % % 
% % % % \begin{frame}[fragile]
% % % %   \begin{block}{Projection Onto First 2 PC's}\pause
% % % % \begin{lstlisting}
% % % % ### SVD manually in serial
% % % % X.tmp <- crossprod(X.std)        # X'X (local)
% % % % X.tmp <- allreduce(X.tmp)
% % % % dim(X.tmp) <- c(p, p)
% % % % ret <- eigen(X.tmp)              # X'X = V D^2 V'
% % % % d <- sqrt(ret$values)
% % % % v <- ret$vectors
% % % % u <- X.std %*% v %*% diag(1/d)   # Why X V D^(-1)) = U?
% % % % \end{lstlisting}
% % % % \end{block}
% % % % \end{frame}
% % % % 
% % % % \subsection{Cluster}
% % % % 
% % % % \begin{frame}[fragile]
% % % %   \begin{block}{Clustering}\pause
% % % % \begin{lstlisting}
% % % % ### Clustering
% % % % set.seed(1234)                  # Set overall seed
% % % % X.kms <- kmeans(X.std, 3)       # K-means
% % % % X.kms
% % % % X.kms.cid <- X.kms$cluster      # Classification
% % % % 
% % % % library(EMCluster)              # Model-based clustering
% % % % X.mbc <- init.EM(X.std, 3)      # Initial by em-EM
% % % % X.mbc
% % % % X.mbc.cid <- X.mbc$class        # Classification
% % % % \end{lstlisting}
% % % % \end{block}
% % % % \end{frame}
% % % % 
% % % % \begin{frame}[fragile]
% % % %   \begin{block}{Cluster Validation}\pause
% % % % \begin{lstlisting}
% % % % ### Validation
% % % % X.kms.adjR <- RRand(X.cid, X.kms.cid)$adjRand       # Adjusted Rand index
% % % % X.mbc.adjR <- RRand(X.cid, X.mbc.cid)$adjRand
% % % % \end{lstlisting}
% % % % \end{block}
% % % % \end{frame}
% % % % 
% % % % \begin{frame}[fragile]
% % % %   \begin{block}{Cluster ID Variable}\pause
% % % % \begin{lstlisting}
% % % % ### Swap classification id
% % % % X.kms.cid[X.kms.cid == 2] <- 4
% % % % X.kms.cid[X.kms.cid == 3] <- 2
% % % % X.kms.cid[X.kms.cid == 4] <- 3
% % % % X.mbc.cid[X.mbc.cid == 2] <- 4
% % % % X.mbc.cid[X.mbc.cid == 3] <- 2
% % % % X.mbc.cid[X.mbc.cid == 4] <- 3
% % % % \end{lstlisting}
% % % % \end{block}
% % % % \end{frame}
% % % % 
% % % % \subsection{Plot}
% % % % 
% % % % \begin{frame}[fragile]
% % % %   \begin{block}{Plot}\pause
% % % % \begin{lstlisting}
% % % % ### Display on first 2 components
% % % % pdf("serial_plot.pdf")
% % % % 
% % % % par(mfrow = c(2, 2))
% % % % plot(X.prj, col = X.cid + 1, pch = X.cid,
% % % %      main = "iris (true)", xlab = "PC1", ylab = "PC2")
% % % % plot(X.prj, col = X.kms.cid + 1, pch = X.kms.cid,
% % % %      main = paste("iris (k-Means)", sprintf("%.4f", X.kms.adjR)),
% % % %      xlab = "PC1", ylab = "PC2")
% % % % plot(X.prj, col = X.mbc.cid + 1, pch = X.mbc.cid,
% % % %      main = paste("iris (Model-based)", sprintf("%.4f", X.mbc.adjR)),
% % % %      xlab = "PC1", ylab = "PC2")
% % % % accuracy <- c(X.kms.adjR, X.mbc.adjR)
% % % % names(accuracy) <- c("k-Means", "Model-based")
% % % % barplot(accuracy, main = "Clustering Accuracy")
% % % % 
% % % % dev.off()
% % % % \end{lstlisting}
% % % % \end{block}
% % % % \end{frame}
% % % % 
% % % % 
% % % % \begin{frame}
% % % %   \begin{block}{Plot}\pause
% % % % \begin{center}
% % % %   \includegraphics[scale=.37]{other/serial_plot.pdf}
% % % % \end{center}
% % % % \end{block}
% % % % \end{frame}


\section[Iris]{In-Depth Example Examining the Iris Dataset with pbdR}

\hidenum
\begin{frame}[noframenumbering]
\frametitle{Contents}
 \tableofcontents[currentsection,hideothersubsections,sectionstyle=show/hide]
\end{frame}
\shownum

\subsection{Examining the Iris Dataset}

\begin{frame}[fragile]
  \begin{block}{The Iris Dataset}\pause
\begin{lstlisting}
rm(list = ls())                    # Clean environment

library(pbdDMAT, quiet = TRUE)     # Load library
init.grid()
if(comm.size() != 4)
  comm.stop("4 processors are required.")

### Load data
X <- as.matrix(iris[, -5])         # Dimension 150 by 4
X.cid <- as.numeric(iris[, 5])     # True id

### Convert to ddmatrix
X.dmat <- as.ddmatrix(X)
\end{lstlisting}
\end{block}
\end{frame}


\begin{frame}[fragile]
  \begin{block}{Standardizing}\pause
\begin{lstlisting}
### Standardized
X.std <- scale(X.dmat)
mu <- as.matrix(colMeans(X.std))
cov <- as.matrix(cov(X.std))
comm.print(mu)
comm.print(cov)
\end{lstlisting}
\end{block}
\end{frame}

\begin{frame}[fragile]
  \begin{block}{Projection Onto First 2 PC's}\pause
\begin{lstlisting}
### SVD
X.svd <- svd(X.std)

### Project on column space of singular vectors
A <- X.svd$u %*% diag(X.svd$d, type="ddmatrix")
B <- X.std %*% X.svd$v            # A ~ B
X.prj <- as.matrix(A[, 1:2])      # Only useful for plot
\end{lstlisting}
\end{block}
\end{frame}

\subsection{Cluster}

\begin{frame}[fragile]
  \begin{block}{Clustering}\pause
\begin{lstlisting}
### Clustering
library(pmclust, quiet = TRUE)
comm.set.seed(123, diff = TRUE)

X.dmat <- X.std
PARAM.org <- set.global.dmat(K = 3)      # Preset storage
.pmclustEnv$CONTROL$debug <- 0           # Disable debug messages
PARAM.org <- initial.center.dmat(PARAM.org)
PARAM.kms <- kmeans.step.dmat(PARAM.org) # K-means
X.kms.cid <- as.vector(.pmclustEnv$CLASS.dmat)
\end{lstlisting}
\end{block}
\end{frame}

\begin{frame}[fragile]
  \begin{block}{Cluster Validation}\pause
\begin{lstlisting}
### Validation
X.kms.adjR <- EMCluster::RRand(X.cid, X.kms.cid)$adjRand
comm.print(X.kms.adjR)
\end{lstlisting}
\end{block}
\end{frame}

\begin{frame}[fragile]
  \begin{block}{Cluster ID Variable}\pause
\begin{lstlisting}
### Swap classification id
tmp <- X.kms.cid
X.kms.cid[tmp == 1] <- 3
X.kms.cid[tmp == 2] <- 1
X.kms.cid[tmp == 3] <- 2
\end{lstlisting}
\end{block}
\end{frame}

\subsection{Plot}

\begin{frame}[fragile]
  \begin{block}{Plot}\pause
\begin{lstlisting}
### Display on first 2 components
if(comm.rank() == 0){
  pdf("dmat_plot.pdf")
  
  par(mfrow = c(2, 2))
  plot(X.prj, col = X.cid + 1, pch = X.cid,
       main = "iris (true)", xlab = "PC1", ylab = "PC2")
  plot(X.prj, col = X.kms.cid + 1, pch = X.kms.cid,
       main = paste("iris (kmeans)", sprintf("%.4f", X.kms.adjR)),
       xlab = "PC1", ylab = "PC2")
  
  dev.off()
}
\end{lstlisting}
\end{block}
\end{frame}


% \begin{frame}
%   \begin{block}{Plot}\pause
% \begin{center}
%   \includegraphics[scale=.37]{other/dmat_plot.pdf}\\
%  \vspace{-1cm}\url{http://www.nics.tennessee.edu/getting-an-allocation} 
% \end{center}
% \end{block}
% \end{frame}
%%%%%%%%%%%%%%%%%%%%%%%%%%%%%%%%%%%%%%%%
%%     Last few slides
%%%%%%%%%%%%%%%%%%%%%%%%%%%%%%%%%%%%%%%%
\section{Wrapup}

\hidenum
\begin{frame}[noframenumbering]
\frametitle{Contents}
 \tableofcontents[currentsection,hideothersubsections,sectionstyle=show/hide]
\end{frame}
\shownum

\begin{frame}
  \begin{block}{Where to Learn More}
    \begin{enumerate}
      \item The \pkg{pbdDEMO} package\\
      \url{http://cran.r-project.org/web/packages/pbdDEMO/}\\
      Vignette: \url{http://goo.gl/eBsIh}
      \item Our Google Group:\\
        \url{http://group.r-pbd.org}
      \item Get an allocation with us!\\
        {\small \url{http://www.nics.tennessee.edu/getting-an-allocation}\\ }
    \end{enumerate}
\end{block}
\end{frame}


\section*{}


% \begin{frame}[fragile,allowframebreaks]
%   \tiny
%   \bibliographystyle{plainnat}
%   \bibliography{pbdbib}
% \end{frame}


\hidenum
\begin{frame}[noframenumbering]
 \begin{block}{Thanks for coming!}
 \begin{center}
     {\Large Questions?  Comments?}\\[.4cm]
     Please help us improve this tutorial by completing a short survey:\\
     \url{http://www.surveymonkey.com/s/W8VLJSP}
  \end{center}
 \end{block}
\end{frame}


\end{document}
%%%%%%%%%%%%%%%%%%%%%%%%%%%%%%%%