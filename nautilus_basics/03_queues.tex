
\lsc{The Job Queue System}\label{jqssec}
Generally, you will do very little direct interaction with Nautilus.  You should not really be using Nautilus in the same way that you use a workstation; it is not a workstation.  Nautilus is a shared system.  No one person is the sole user of Nautilus at any given time (usually).  \\\\
%
The way you will want to run your analyses is by submitting your job to a queue system on Nautilus.  This system handles all of the user requests and keeps the various demands for resources (cores, ram) from different jobs from stepping on each others toes, so to speak.\\\\
%
A very thorough explanation of the various options and ways of interacting with the job queue system is provided at the \href{http://www.nics.tennessee.edu/computing-resources/nautilus/Batch_Scripts}{batch scripts page on the NICS site}, and so we will not reproduce that information here.  We merely provide a quick sketch of the general process.  For full details and explanations, see the NICS site.  However, you can find some example scripts in the Examples section, Section \ref{egsec}.\\\\
%
However, there are a few options which you should be made aware of.  First, processor allocations are given by node, not by core.  On Nautilus, there are 8 cores per node, so when requesting processors for your job, you should generally request multiples of 8 (because that is what you are going to get anyway).  If you request 1 core, you will get 8; requesting 9 gets you 16, etc.  Likewise, since allocations are by node, ram is allocated similarly.  Each node has 4gb of ram per cpu, and so if you request 8 cores, you will get 32gb of ram.\\\\
%
As for dealing with the queue system, the first step is usually to create an appropriate batch file for your particular job.  This batch file will contain the information Nautilus needs to run your job.  The kind of information you will need will depend on what kind of job you wish to submit.  See the examples in section \ref{egsec} as well as the \href{http://www.nics.tennessee.edu/computing-resources/nautilus/Batch_Scripts}{batch scripts page on the NICS site}.\\\\
%
Once you have your job file prepared, you can submit it to the queue using the \texttt{qsub} command.  So if you have a job file called \texttt{jobfile.pbs} prepared and you are in the directory of this file, then you can submit it to the queue system via the command
\begin{lstlisting}[language=sh]
qsub jobfile.pbs 
\end{lstlisting}%%%
You can check on your job via the commands \texttt{showq} and \texttt{qstat}, but will probably want to use the job file options to send out emails to you.\\\\
%
Finally, if you ever need to prematurely kill a job (whether it is running or merely queued), you can do so with the \texttt{qdel} command.