\lsc{R on Nautilus}
Performing an analysis with R on a remote system can be challenging at first.  The aim of this section is to help with the transition of specifically running R for analysis on a local desktop system to running R on a remote parallel system.

\lsuc{Packages and Libraries}

\subsubsection{Using the Existing Library}
Installing packages, especially complicated ones like Rmpi and gputools, can be difficult on a remote system, especially for the R user who may have never had to deal with compiling things before.  Thankfully, many of the packages you will want to use are already installed for you.  If we are using R version 2.12.0, then the default library path is
\begin{lstlisting}[language=sh]
/sw/analysis/r/2.12.0/sles11.1\_intel11.1/R-2.12.0/library} 
\end{lstlisting}%%%
So merely loading R via the module system and then, for instance, if you want to use the \texttt{multicore} library, since this is already in the default library, all you have to do is
\begin{lstlisting}[language=rr]
library(multicore)
\end{lstlisting}
If you require an additional package which is not currently installed, the simplest way to get this is to fill out a request at the \href{http://www.nics.tennessee.edu/software-request}{Request Software Installation page at the NICS site}.

\subsubsection{Managing Your Own R Library}
You can also of course manage your own R library, though for most this will at best be unnecessary, and at worst a needless, frustrating exercise.  If you do need to manage, at least in part, your own R library, then you must install packages from source, either by downloading the source package (say with the utility wget) and install it with the command
\begin{lstlisting}[language=sh]
R CMD INSTALL [options] packages
\end{lstlisting}%%%
or by using the \texttt{install.packages()} command in an interactive R session, setting the various options as needed (\texttt{lib=}, \texttt{INSTALL\_opts=}, \dots).  One option that is not optional (without overriding an R environment variable; see paragraph to follow for details) when installing a package to a custom, user-managed library is the \texttt{lib=} argument.  This is so because you do not have write access to your default library.  Similarly, to load a package from a custom library, make sure you set the appropriate \texttt{lib.loc=} argument when issuing the \texttt{library()} command in R.\\\\
%
One additional note to consider when maintaining your own R package library is that you will likely need to manually set a few R environment variables in order to get some packages to install, or even load.  For a mostly complete list of R environment variables, enter the command \texttt{help("environment variables")} in an interactive R session.

\lsuc{Different R Versions}
For more information, see the \href{http://www.nics.tennessee.edu/computing-resources/nautilus/software?&software=r}{R page at the NICS site}.

\lsuc{Running Batch Jobs}
As mentioned in Section \ref{jqssec}, you generally should not be running R in quite the same way that you would run it on your workstation.  On your workstation, you probably use R interactively; that is, you load up an R session and submit lines to the R terminal as you need. \\\\
%
By contrast, on Nautilus, you will be running R scripts in batch.  That is, you will not start an interactive R session.  Instead, you will issue a command from the terminal (really, from you job script) to start R, run your analysis, direct output to a file, and kill R when the script completes (or encounters an error).  Now, for the purposes of appropriate resource allocation, this should be be done in your jobfile that you submit via qsub, rather than directly handled in the terminal.  This is merely intended to illustrate what is actually going on.  \\\\
%
One way to do this is to issue the terminal command \texttt{R CMD BATCH}.  So say the script you wish to have R run is called \texttt{myscript.R}.  Then issuing the command
\begin{lstlisting}[language=sh]
R CMD BATCH myscript.R 
\end{lstlisting}%%%
will run the \texttt{myscript.R} script through R in a way that is somewhat equivalent to starting an interactive R session and issuing the R commands 
\begin{lstlisting}[language=rr]
sink("myscript.Rout")
source("myscript.R")
q(save="no")
\end{lstlisting}

% \lsuc{Starting an interactive R session}
% Most of the time when you use R on Nautilus, you should submit a batch job via the queue system.  However, on occasion you may need to run R interactively (like you would on your own computer).  
