\lsc{Interacting with Nautilus}
A complete set of documentation on connecting to Nautilus can be found at the \href{http://www.nics.tennessee.edu/getting-started/access}{the access page on the NICS site}.  We will provide some examples that should illustrate the process sufficiently for users, but if there is ambiguity, you are always encouraged to check the \href{http://www.nics.tennessee.edu}{NICS site} documentation.

\lsuc{Connecting to Nautilus}
In order to connect to and use Nautilus, you must have a \textbf{S}ecure \textbf{SH}ell (SSH) client.  SSH is a protocol that allows encrypted connections between remote computers.  It consists of two parts, the ssh server (Nautilus) and the ssh client (your computer).

\subsubsection{Connecting from Mac and Linux}\label{confml}
Mac OS X and Linux have a ssh client bundled with the operating system.  On a Mac, you can open a terminal by navigating to your Applications folder, then the Utilities folder, and running the app ``Terminal''.  On Linux, the method for starting a terminal will depend to some degree on your distribution of choice.\\\\
%
If you would prefer, you can also use Putty (see section \ref{confromwin} below) as it is multiplatform (although this is not necessary).  It provides a graphical user interface for starting (but not using) a terminal session.  For the remainder, we will not assume that you are using Putty (though if you are, it should be reasonably clear how to take the information here and use it appropriately in Putty; alternatively, see section \ref{confromwin} below).\\\\
%
For demonstration purposes, say your username for your Nautilus account is \texttt{nautuser}.  Exactly how you connect will depend on whether you have an XSEDE account or a Director's Discretion account from section \ref{getacct}.  If you have an XSEDE account, then you would enter into the terminal:
\begin{lstlisting}[language=sh]
ssh nautuser@login.nautilus.nics.xsede.org
\end{lstlisting}%%%
On the other hand, if you have a Director's Discretion account, you would login by entering into the terminal:
\begin{lstlisting}[language=sh]
ssh nautuser@login.nautilus.nics.tennessee.edu
\end{lstlisting}%%%

Of course, it is inconvenient to type this into the terminal every time, so you may wish to configure a ``shortcut'' of sorts.  To do this, you would need to create a file called \texttt{config} and put that in the \texttt{.ssh/} (note the preceding dot) subdirectory of your home directory.  If this directory does not exist, create it.  With your text editor of choice, you might enter into this config file:\\\\
%
\texttt{ Host nautilus\\
HostName login.nautilus.nics.xsede.org\\
User nautuser\\
TCPKeepAlive yes\\
}

filling in the appropriate information as needed, so that to connect you need only type
\begin{lstlisting}[language=sh]
ssh nautilus
\end{lstlisting}%%%
into the terminal.\\\\
%
However you choose to go about it, begin an ssh session with Nautilus.  On your first login, you will be prompted about adding a RSA key to your cache.  Do so.


See section \ref{nautpw} below for complete details about your password.

\subsubsection{Connecting from Windows}\label{confromwin}
Windows users will have to download an ssh client, such as \href{http://www.chiark.greenend.org.uk/~sgtatham/putty/}{Putty}.  For a complete set of Putty documentation, see \href{http://the.earth.li/~sgtatham/putty/0.62/htmldoc/}{the putty user manual}.\\\\
%
After starting putty, you will need to enter the Nautilus host name under the Session category.  Make sure that the connection type is set to SSH and that the port is 22.  Which host name you use will depend on which type of account you created in section \ref{getacct}.  If you have an XSEDE account, then you would use the hostname\\\\
%
\texttt{login.nautilus.nics.xsede.org}
\\\\
On the other hand, if you have a Director's Discretion account, you would login by entering into the terminal:\\\\
%
\texttt{login.nautilus.nics.tennessee.edu}
\\\\
You can save this session either as the default or under the name of your choosing.  There are many configurable options available to you in Putty, but they are not needed to proceed (though altering them may make your experience more enjoyable).\\\\
%
Your first time connecting you will be greeted with a warning about a secure RSA key.  Select the option to add the key to Putty's cache.  You will then be greeted by a prompt asking what name you wish to login as.  Enter the account name from the account creation step and press enter.  You will then be asked to enter your password.  See section \ref{nautpw} below for complete details about your password.

\subsubsection{Your Nautilus Password}\label{nautpw}
Your password for the Nautilus system will consist of two pieces:  your PIN that you set in the account creation process, and the number that shows on your NICS token.  For illustration, say your PIN is 1234 (do not make this your PIN) and your NICS token reads 567 890.  Then the password you would enter when prompted would be 1234567890.\\\\
%
The number displayed on your NICS token will change roughly every 30 seconds (the little bars on the lower left will give you a sense for how much longer you have with that particular set of digits).  For more information, see the \href{http://www.nics.tennessee.edu/getting-started/access#OTPAuthentication}{access page on the NICS site}.


\lsuc{Basic Commands}
Once you are connected to Nautilus, you will be greeted by a text prompt.  If this is your first interaction with such a system, you may have no idea how to proceed.  This part can be confusing at first, but with a little time and patience, it will become second nature to you.\\\\
%
When interacting with the shell, never forget that it is case sensitive.  The table below
\begin{table}[h]
 \centering
  \begin{tabular}{lll}\hline
   command & description & example\\\hline
   ls & list files in a directory & ls\\
   mkdir & make directory & mkdir dir\\
   cd & change directory & cd dir\\
   cp & copy a file & cp a b\\
   mv & move and/or rename a file & mv a dir\\
   man & the help system & man mv\\\hline       
  \end{tabular}
\end{table}  
provides the basic commands you will need most of the time.\\\\
%
Never be afraid to read the man pages.  If you ever find yourself wondering ``is it possible to'', the answer is yes, and it's probably an option explained in the man page for the program.  To search within a man page, use the \texttt{/} key followed by your query.  If there are multiple hits for your query, you can jump to the next one by pressing \texttt{n} and to the previous one with \texttt{N}.

  \lsuc{Loading Software}
Much of the software available on Nautilus must first be loaded by the user before it can be used.  If you enter the command
\begin{lstlisting}[language=sh]
R
\end{lstlisting}%%%
into the terminal (remember, case sensitive), you will get the message
\begin{lstlisting}[language=sh]
-bash:  R:  command not found
\end{lstlisting}%%%
assuming your shell is bash.  At any time you can see which shell you are using by entering the command
\begin{lstlisting}[language=sh]
file /bin/sh
\end{lstlisting}%%%
So how do we start R?  You must first load R through the module system.  To do so, you would enter the command
\begin{lstlisting}[language=sh]
module load r
\end{lstlisting}%%%
into the terminal, and R will now start as expected after entering the ``R'' command into the terminal.  You only have to use the module load command once per login session (but you must load R again after a disconnect).  
\begin{lstlisting}[language=sh]
module load r
R
\end{lstlisting}%%%`
The module system is very powerful and very important to your life on this (or any other) supercomputer.  Entering the command
\begin{lstlisting}[language=sh]
man module
\end{lstlisting}%%%
will give you a complete description of the functionality and use of the module system.  For more information, see the \href{http://www.nics.tennessee.edu/computing-resources/nautilus/software}{Nautilus software page on the NICS site}.\\\\
%
The table below summarizes the use of the module system.

\begin{table}[h]
 \centering
\begin{tabular}{ll}\hline
Command & Purpose\\\hline
module avail & List available programs in the module system\\
module load $<$program$>$ & Load program\\
module unload $<$program$>$ & Unload program\\\hline
% module swap $<$first$>$ $<$second$>$ & Swap first program for second (in cases of confli
\end{tabular}
\end{table}

\lsuc{Choosing a Text Editor}
Choosing a text editor can be a very lengthy journey, somewhat akin to seeking out religious enlightenment.  No one text editor is perfect, and a discussion of the features of common text editors is well beyond the scope of this document.  Eventually, you will likely want to choose use either \href{https://www.gnu.org/software/emacs/}{Emacs} (no relation to Apple) or \href{http://www.vim.org/}{vim}.  Each of these is available on Nautilus each time you log in (you do not have to load them through the module system).  If you are not familiar with either of these editors, they can be very difficult to learn at first, but learning to use one of these well is a good idea.\\\\
%
For now, these might be a bit intimidating and act as one more hurdle in getting to the real work you want to do.  To that end, it might be a good idea to start with a more friendly text editor, such as nano.  Nano is not loaded by default each session, so you will have to enter the command
\begin{lstlisting}[language=sh]
module load nano
\end{lstlisting}%%%
to first load nano, and then call the editor by entering
\begin{lstlisting}[language=sh]
nano
\end{lstlisting}%%%
The commands for nano are visible at the bottom of the screen.  So here, Ctrl together with R reads in a file, Ctrl together with O saves, etc.  The editor is extremely minimalistic in terms of features, but is a fine place to start until you become more comfortable working in a terminal. \\\\
If you wish to have nano loaded each time you log in to Nautilus, you could start nano, enter Ctrl+R to bring up the read file dialogue.  For the file, enter
\begin{lstlisting}[language=sh]
~/.bashrc
\end{lstlisting}%%%
and somewhere inside this configuration file, add the line
\begin{lstlisting}[language=sh]
module load nano
\end{lstlisting}%%%
Press Ctrl+O to save the edit, and whenever you log in from now on, you do not first have to first load nano through the module system to be able to run nano.

\lsuc{Transferring Files}
To transfer files over to Nautilus, you have a variety of options, explained in depth at the \href{http://www.nics.tennessee.edu/computing-resources/data-transfer}{Data Transfer page at the NICS site}.  For getting started, probably the two methods of file transfer available that will be of interest are sftp for small files and GridFTP for big files.

\subsubsection{Small File Transfer}
If the files you wish to transfer are not particularly large (eg, small datasets, some R scripts, etc.), probably the easiest way to proceed is to connect to Nautilus by sftp (the secure file transfer protocol).  \\\\
%
Mac and Linux users can use the terminal as with ssh, and Windows users can use Putty (via PSFTP) to connect via sftp.  Although there is gui program available for managing files over sftp called \href{http://filezilla-project.org/}{filezilla}.  If you elect to use filezilla, you should read its documentation, although the program should be fairly self-explanatory.\\\\
%
To connect from a terminal (Mac/Linux), you can enter the command \texttt{sftp} as you would use \texttt{ssh} in Section \ref{confml}.  So for example, you might enter the command
\begin{lstlisting}[language=sh]
sftp nautuser@login.nautilus.nics.xsede.org
\end{lstlisting}%%%
or if you set up your ssh config file as in the example, you could do
\begin{lstlisting}[language=sh]
sftp nautilus
\end{lstlisting}%%%
From here, you can transfer files via the self-explanatory commands \texttt{put} and \texttt{get}.  The above applies for Windows users, after you replace ``sftp'' with ``psftp''.\\\\
%
Your sftp password is the same as your ssh password. 

\subsubsection{Large File Transfer}
If you need to transfer large files, such as your full dataset, using sftp is not recommended.  In this case, the preferred method is GridFTP.  You will need a special GridFTP client such as \href{https://www.globusonline.org/}{globus-url-copyh} or \href{http://dims.ncsa.illinois.edu/set/uberftp/}{uberftp}.  For details about how to set up and use GridFTP with these clients, see the \href{http://www.nics.tennessee.edu/user-support/general-support/data-transfer/gridftp}{GridFTP page on the NICS site}.
