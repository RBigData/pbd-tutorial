\section{Introduction}
% The skills required to be a good researcher in statistics and a good programmer are not necessarily complimentary.  

Herein, we hope to ease the transition for R users moving from the desktop to the \href{http://www.nics.tennessee.edu/computing-resources/nautilus}{Nautilus supercomputer}.  Knowledge of R as well as some basic principles of parallelism are assumed; however, no particular mastery of various parallel R API's is assumed.  \\\\
%
% Throughout, we assume that you are familiar with R.
%
This document is ideally suited for the user who has only or mostly done R development on a single, standalone workstation (laptop/desktop computer).  This is by no means meant to be complete or comprehensive, and is primarily focused on bridging the initial gap of knowledge in moving from a workstation to a remote system for the R user.

  \lsuc{What is Nautilus?}

Nautilus, an SGI Altix UV 1000 system, is the centerpiece of NICS \href{http://rdav.nics.tennessee.edu/}{Remote Data Analysis and Visualization Center} (RDAV). It has 1024 cores (Intel Nehalem EX processors), 4 terabytes of global shared memory, and 8 GPUs in a single system image. Nautilus currently has a 427 TB Lustre file system, a CPU speed of 2.0 GHz, and a peak performance of 8.2 Teraflops.\\\\
%
The primary purpose of Nautilus is to enable data analysis and visualization of data from simulations, sensors, or experiments. Nautilus is intended for serial and parallel visualization and analysis applications that take advantage of large memories, multiple computing cores, and multiple graphics processors. Nautilus allows for both utilization of a large number of processors for distributed processing and the execution of legacy serial analysis algorithms for very large data processing by large numbers of users simultaneously.   

  \lsuc{Getting an Account}\label{getacct}
See the \href{http://www.nics.tennessee.edu/getting-started/accounts}{allocations and accounts page on the NICS site} for details.