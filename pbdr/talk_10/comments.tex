\documentclass{article}
\usepackage{color}
\begin{document}
\definecolor{auburn}{rgb}{0.43, 0.21, 0.1}
\section{Notes for pbdR: R for HPC Data Analysis}
\begin{enumerate}
\item The pbdR Core Team
  \begin{itemize}
  \item This is the team that built and continues to build the pbdR software
  \end{itemize}
\item Overview: Three parts
  \begin{enumerate}
  \item Brief introduction to R (mostly where to get help)
  \item The pbdR infrastructure for HPC
  \item The Demos
  \end{enumerate}
\item What is R?
  \begin{itemize}
  \item Built for interactive data analysis
  \item Has roots in S laguage from Bell Labs in the 1970's
  \item Extensible: The most comprehensive and cutting-edge collection
    of techniques
  \end{itemize}
\item Who uses R?
  \begin{itemize}
  \item Microsoft bought Revolution Analytics
  \item R Consortium recently formed: https://www.r-consortium.org/about
  \end{itemize}
\item Learning R?
  \begin{itemize}
  \item Many books and resources
  \item {\em aRrgh} is a funny one you may enjoy if with a CS background
  \end{itemize}
\item Other resources?
  \begin{itemize}
  \item Task Views: give a great overview of the diversity in R
  \item Stackoverflow is a great source of specific help
  \end{itemize}
\item pbdR: Sustainble Path, Infrastructure overview
  \begin{itemize}
  \item {\bf Linear algebra} that powers PDE solvers in simulation science
    is the same math under much of data analysis
  \item Excellent {\bf scalable librarie}s exist and continue to be
    developed
  \item Picture: {\bf Three flavors} of parallel hardware: distributed memory,
    shared memory, and co-processor
  \item Libraries are placed on the hardware they use
  \item {\color{blue}Free libraries are blue}, {\color{auburn}vendor libraries are brown}
  \item On the right are profiling libraries and I/O libraries
  \item pbdR packages are indicated in blue bubbles, dashed lines
    under development
  \item The pbdDEMO package is a good place to start learning pbdR. It
    contains a textbook-style vignette tutorial.
  \item We first talk about pbdMPI
  \end{itemize}
\item pbdMPI
  \begin{itemize}
  \item pbdMPI is a high-level interface to MPI. The object oriented
    code handles object details. Rmpi is an older package that is
    closer to the low level API of MPI.
  \end{itemize}
\item pbdMPI
  \begin{itemize}
  \item An example where the machine knows more than you do so it's
    preferable to let it do the work.
  \end{itemize}
\item SPMD pbdMPI
  \begin{itemize}
  \item An example SPMD script that can implement a simplified
    MapReduce. Scripts are executed with mpirun.
  \end{itemize}
\item pbdDMAT
  \begin{itemize}
  \item ddmatrix methods map data onto a processor grid.
  \end{itemize}
\item pbdDMAT
  \begin{itemize}
  \item An example layout of a 9 x 9 block-cyclic matrix with a 2 x 2
    block size on a 2 x 3 processor grid.
  \item Easy methods are available to intuitively manage such layouts.
  \end{itemize}
\item pbdDMAT
  \begin{itemize}
  \item The local view of the 9 x 9 matrix.
  \end{itemize}
\item pbdDMAT
  \begin{itemize}
  \item The power of the ddmatrix class is {\bf no change in syntax} for
    over 100 functions.
  \item as.rowblock etc. intuitively manage data redistributions.
  \end{itemize}
\item Truncated SVD
  \begin{itemize}
  \item An example of fast prototyping of code from a journal.
  \item Took an evening of reading the paper, then an hour to have
    serial code working.
  \end{itemize}
\item Truncated SVD
  \begin{itemize}
  \item Parallel version has very few changes when working with ddmatrix.
  \end{itemize}
\item Truncated SVD
  \begin{itemize}
  \item Scaling results in one day!
  \end{itemize}
\item Data Input
  \begin{itemize}
  \item Parallel reading of data is the hardest part.
  \item Read the most natural way, then redistribute.
  \item Binary formats are best, pbdADIOS and pbdNCDF4 are available.
  \end{itemize}
\item Now the Demos
\item Where to learn more?
\end{enumerate}
\section{Running pbdR Example Scripts}
\begin{verbatim}
$ ls scripts
0_mpirun        3_mcsim.r       5b_cov_2.r      8a_rf_fgl.r
1_hello.r       4_jid.r         6_ols.r         8b_rf_Boston.r
2_gt.r          5a_cov_1.r      7_pca.r         9_cluster.r
\end{verbatim}
Each of these scripts can be run as shown in {\tt 0\_mpirun}.
\begin{verbatim}
$ cat 0_mpirun
mpirun -np 4 Rscript _scriptfile_
\end{verbatim}
Scripts 8 and 9 will require installing two more R packages in the
virtual machine image as follows. The command will pop up a list to
choose a mirror site. Choose one close to your location.
\begin{verbatim}
$ R
> install.packages("randomForest")
> install.packages("pmclust")
\end{verbatim}

\paragraph{1\_hello.r} A "Hello World" with some printing function
details.

\paragraph{2\_gt.r} High-level basic MPI use and random number
generation.

\paragraph{3\_mcsim.r} The standard $\pi$ estimation via parallel
simulation on a circle and a square.

\paragraph{4\_jid.r} The {\tt get.jid( )} function helps splitting
work among processors.

\paragraph{5a\_cov\_1.r} Computes a covariance matrix to illustrate
low-level tools that may be needed when building custom scripts.

\paragraph{5a\_cov\_1.r} Computes a covariance matrix the easy way
using ddmatrix class.

\paragraph{6\_ols.r} Computes least squares regression via normal
equations to illustrate low-level tools that may be needed when
building custom scripts. For fitting linear models the easy and more
numerically accurate way, there is a function lm.fit( ) that uses the
ddmatrix class.

\paragraph{7\_pca.r} Computes principal components decomposition of a
random matrix and figures out how many components are needed to
represent 90\% of variability. Because the matrix is random, thei
should be just under 90\% of the columns.

\paragraph{8a\_rf\_fgl.r} Random forest classification of the Forensic
Glass Data set. Forest is generated in parallel, the classifier votes
are combined, then used to classify new data in parallel (the example
script actually classifies the same data).

\paragraph{8b\_rf\_Boston.r} Random forest regression on Boston housing
data. Proceeds as the script above. Predicts housing value.

\paragraph{9\_cluster.r} Model based clustering (fits a Gaussian
mixture model) and also computes k-means clustering (the special case
of identically spherical Gaussians). Uses ddmatrix class to manage
distributed data.

\section{Running ADIOS/pbdR Heat Transfer Example}
From virtual machine home directory:
\begin{verbatim}
$ cd Tutorial/heat_transfer
$ mpirun -np 12 ./heat_transfer_adios1 heat  3 4    50 40   20 500
$ cd R
$ mpirun -np 4 Rscript test_heat.r
\end{verbatim}
To view the resulting png file sequences, which are split into four pieces on output:
\begin{verbatim}
$ eog *_0.png
$ eog *_1.png
$ eog *_2.png
$ eog *_3.png
\end{verbatim}
The image is split because R does not have distributed graphics. A
connection via VisIt is partially done and may not be far in the
future. Currently pbdR can accept a communicator when running in situ
with other software so the data access need not involve moving data.
\end{document}