\section{pbdR}
%\makesubcontentsslides

\subsection{The Project}

\begin{frame}{\pbdR Interfaces to Libraries: Sustainable Path}
  \vspace{-1ex}
  \centering\includegraphics[trim=0cm 5cm 0cm 3cm,clip=true,width=0.85\textwidth]
  {../common/pics/hardware/ParallelHardware27.pdf}
  \scriptsize
  \begin{block}{Why use HPC libraries?}
    \begin{itemize}[<+-|alert@+>]
    \item Data analysis uses much of the same basic math as simulation science.
    \item The libraries represent 30+ years of parallel algorithm research
    \item \emph{They're tested.} \emph{They're
        fast.}  \emph{They're scalable.}
    \item Many science communities are invested in their API.
    \end{itemize}
  \end{block}
\end{frame}

\subsection{pbdMPI}

\begin{frame}
  \begin{block}{pbdMPI: a High Level Interface to MPI}
    \begin{itemize}
    \item API is simplified: defaults in control objects.
    \item S4 methods: extensible to complex \R objects.
    \item Additional error checking
    \item Array and matrix methods without serialization: faster than
      \pkg{Rmpi}.
    \end{itemize}
    \begin{center}
      \vspace{0.2cm}\scriptsize
      \begin{tabular}{ll} \hline\hline
        \pkg{pbdMPI} (S4) & \pkg{Rmpi}                \\ \hline
        \code{\color{blue}allreduce}    & \code{mpi.allreduce}      \\
        \code{\color{blue}allgather}    & \code{mpi.allgather},
        \code{mpi.allgatherv},
        \code{mpi.allgather.Robj} \\
        \code{bcast}        & \code{mpi.bcast},
        \code{mpi.bcast.Robj}     \\
        \code{gather}       & \code{mpi.gather},
        \code{mpi.gatherv},
        \code{mpi.gather.Robj}    \\
        \code{recv}         & \code{mpi.recv},
        \code{mpi.recv.Robj}      \\
        \code{reduce}       & \code{mpi.reduce}         \\
        \code{scatter}      & \code{mpi.scatter},
        \code{mpi.scatterv},
        \code{mpi.scatter.Robj}   \\
        \code{send}         & \code{mpi.send},
        \code{mpi.send.Robj}      \\ \hline \hline
      \end{tabular}
    \end{center}
  \end{block}
\end{frame}

% \begin{frame}[fragile]{SPMD: Run Many Copies of One Code}
%   \begin{exampleblock}{SPMD ``Hello World:'' a simple ``map\_reduce'' to all}
%     \vspace{-1.5ex}
%     \centering
%     \begin{lstlisting}[title=map\_reduce.r]
% library(pbdMPI, quiet = TRUE)
% init()

% ## Your "Map" code
% n <- comm.rank() + 1

% ## Now "Reduce" and give the result to all
% all_sum <- allreduce(n) # Sum is default

% text <- paste("Hello: n is", n, "sum is", all_sum )
% comm.print(text, all.rank=TRUE)

% finalize()
%     \end{lstlisting}
%     \vspace{-4.5ex}
%     \begin{columns}[t,onlytextwidth]
%       \begin{column}{0.54\textwidth}
%         \begin{lstlisting}[backgroundcolor=\color{white},keywordstyle=\color{black},
% title=Execute this batch script via:]
% mpirun -np 2 Rscript map-reduce.r
%         \end{lstlisting}
%       \end{column}
%       \hfill
%       \begin{column}{0.46\textwidth}
%         \begin{lstlisting}[title=Output:]
% COMM.RANK = 0
% [1] "Hello: n is 1 sum is 3"
% COMM.RANK = 1
% [1] "Hello: n is 2 sum is 3"
%         \end{lstlisting}
%       \end{column}
%     \end{columns}
%   \end{exampleblock}
% \end{frame}



\subsection{pbdDMAT (and pbdBASE, pbdSLAP)}
\newcommand{\Sp}{\hspace{3pt}}
\begin{frame}{Matrix and Vector Operations for Analytics \hfill pbdDMAT}
  \begin{block}{Powered by ScaLAPACK, PBLAS, and BLACS (MKL, SciLIB,
      or ACML)}
    \begin{itemize}
    \item Block-cyclic data layout for scalability and efficiency
    \item No change in R syntax
    \item High-level convenience for data redistributions
      \begin{itemize}
      \item Row-major data: read row-block then convert to block-cyclic
      \item Column-major data: read column-block then convert to block-cyclic
      \end{itemize}
    \end{itemize}
  \end{block}
  \begin{block}{Global and local views of block-cyclic on a 2 $\times$ 3
      processor grid}
\tiny
\begin{minipage}{4.4cm}
$
\left[
      \begin{array}{@{\Sp}l@{\Sp}l@{\Sp}|@{\Sp}l@{\Sp}l@{\Sp}|@{\Sp}l@{\Sp}l@{\Sp}|@{\Sp}l@{\Sp}l@{\Sp}|@{\Sp}l@{\Sp}}
      \color{g11}x_{11} & \color{g11}x_{12} & \color{g12}x_{13} & \color{g12}x_{14} & \color{g13}x_{15} & \color{g13}x_{16} & \color{g11}x_{17} & \color{g11}x_{18} & \color{g12}x_{19}\\
      \color{g11}x_{21} & \color{g11}x_{22} & \color{g12}x_{23} & \color{g12}x_{24} & \color{g13}x_{25} & \color{g13}x_{26} & \color{g11}x_{27} & \color{g11}x_{28} & \color{g12}x_{29}\\\hline
      \color{g21}x_{31} & \color{g21}x_{32} & \color{g22}x_{33} & \color{g22}x_{34} & \color{g23}x_{35} & \color{g23}x_{36} & \color{g21}x_{37} & \color{g21}x_{38} & \color{g22}x_{39}\\
      \color{g21}x_{41} & \color{g21}x_{42} & \color{g22}x_{43} & \color{g22}x_{44} & \color{g23}x_{45} & \color{g23}x_{46} & \color{g21}x_{47} & \color{g21}x_{48} & \color{g22}x_{49}\\\hline
      \color{g11}x_{51} & \color{g11}x_{52} & \color{g12}x_{53} & \color{g12}x_{54} & \color{g13}x_{55} & \color{g13}x_{56} & \color{g11}x_{57} & \color{g11}x_{58} & \color{g12}x_{59}\\
      \color{g11}x_{61} & \color{g11}x_{62} & \color{g12}x_{63} & \color{g12}x_{64} & \color{g13}x_{65} & \color{g13}x_{66} & \color{g11}x_{67} & \color{g11}x_{68} & \color{g12}x_{69}\\\hline
      \color{g21}x_{71} & \color{g21}x_{72} & \color{g22}x_{73} & \color{g22}x_{74} & \color{g23}x_{75} & \color{g23}x_{76} & \color{g21}x_{77} & \color{g21}x_{78} & \color{g22}x_{79}\\
      \color{g21}x_{81} & \color{g21}x_{82} & \color{g22}x_{83} & \color{g22}x_{84} & \color{g23}x_{85} & \color{g23}x_{86} & \color{g21}x_{87} & \color{g21}x_{88} & \color{g22}x_{89}\\\hline
      \color{g11}x_{91} & \color{g11}x_{92} & \color{g12}x_{93} & \color{g12}x_{94} & \color{g13}x_{95} & \color{g13}x_{96} & \color{g11}x_{97} & \color{g11}x_{98} & \color{g12}x_{99}\\
      \end{array}
\right]_{9\times 9}$
\end{minipage} \hspace{4em}
\begin{minipage}{6.5cm}
$\left[
      \begin{array}{@{\Sp}l@{\Sp}l@{\Sp}|@{\Sp}l@{\Sp}l@{\Sp}}
      \color{g11}x_{11} & \color{g11}x_{12} & \color{g11}x_{17} & \color{g11}x_{18}\\
      \color{g11}x_{21} & \color{g11}x_{22} & \color{g11}x_{27} & \color{g11}x_{28}\\\hline
      \color{g11}x_{51} & \color{g11}x_{52} & \color{g11}x_{57} & \color{g11}x_{58}\\
      \color{g11}x_{61} & \color{g11}x_{62} & \color{g11}x_{67} & \color{g11}x_{68}\\\hline
      \color{g11}x_{91} & \color{g11}x_{92} & \color{g11}x_{97} & \color{g11}x_{98}\\
      \end{array}
\right]_{5\times 4}$\hspace{-1.4ex}
$\left[
      \begin{array}{@{\Sp}l@{\Sp}l@{\Sp}|@{\Sp}l@{\Sp}}
      \color{g12}x_{13} & \color{g12}x_{14} & \color{g12}x_{19}\\
      \color{g12}x_{23} & \color{g12}x_{24} & \color{g12}x_{29}\\\hline
      \color{g12}x_{53} & \color{g12}x_{54} & \color{g12}x_{59}\\
      \color{g12}x_{63} & \color{g12}x_{64} & \color{g12}x_{69}\\\hline
      \color{g12}x_{93} & \color{g12}x_{94} & \color{g12}x_{99}\\
      \end{array}
\right]_{5\times 3}$ \hspace{-2ex}
$\left[
      \begin{array}{@{\Sp}l@{\Sp}l@{\Sp}}
      \color{g13}x_{15} & \color{g13}x_{16}\\
      \color{g13}x_{25} & \color{g13}x_{26}\\\hline
      \color{g13}x_{55} & \color{g13}x_{56}\\
      \color{g13}x_{65} & \color{g13}x_{66}\\\hline
      \color{g13}x_{95} & \color{g13}x_{96}\\
      \end{array}
\right]_{5\times 2}$
\\
$\left[
      \begin{array}{@{\Sp}l@{\Sp}l@{\Sp}|@{\Sp}l@{\Sp}l@{\Sp}}
      \color{g21}x_{31} & \color{g21}x_{32} & \color{g21}x_{37} & \color{g21}x_{38}\\
      \color{g21}x_{41} & \color{g21}x_{42} & \color{g21}x_{47} & \color{g21}x_{48}\\\hline
      \color{g21}x_{71} & \color{g21}x_{72} & \color{g21}x_{77} & \color{g21}x_{78}\\
      \color{g21}x_{81} & \color{g21}x_{82} & \color{g21}x_{87} & \color{g21}x_{88}\\
      \end{array}
\right]_{4\times 4}$ \hspace{-2ex}
$\left[
      \begin{array}{@{\Sp}l@{\Sp}l@{\Sp}|@{\Sp}l@{\Sp}}
      \color{g22}x_{33} & \color{g22}x_{34} & \color{g22}x_{39}\\
      \color{g22}x_{43} & \color{g22}x_{44} & \color{g22}x_{49}\\\hline
      \color{g22}x_{73} & \color{g22}x_{74} & \color{g22}x_{79}\\
      \color{g22}x_{83} & \color{g22}x_{84} & \color{g22}x_{89}\\
      \end{array}
\right]_{4\times 3}$ \hspace{-2ex}
$\left[
      \begin{array}{@{\Sp}l@{\Sp}l@{\Sp}}
      \color{g23}x_{35} & \color{g23}x_{36} \\
      \color{g23}x_{45} & \color{g23}x_{46} \\\hline
      \color{g23}x_{75} & \color{g23}x_{76} \\
      \color{g23}x_{85} & \color{g23}x_{86} \\
      \end{array}
\right]_{4\times 2}$
\end{minipage}
\end{block}
\end{frame}

\begin{frame}[fragile]
  \begin{block}{\pbdR\ No change in syntax. \hfill + Data redistribution functions.}
\vspace{-2ex}
  \begin{lstlisting}
x <- x[-1, 2:5]
x <- log(abs(x) + 1)
x.pca <- prcomp(x)
xtx <- t(x) %*% x
ans <- svd(solve(xtx))
  \end{lstlisting}
\vspace{-2ex}
  \begin{center}
  \emph{The above (and over 100 other functions) runs on 1 core with R \\
    or 10,000 cores with \pbdR ddmatrix class}
  \end{center}
\vspace{-2ex}
\begin{lstlisting}
> showClass("ddmatrix")
Class "ddmatrix" [package "pbdDMAT"]
Slots:
Name:     Data     dim    ldim   bldim   ICTXT
Class:  matrix numeric numeric numeric numeric
\end{lstlisting}
\vspace{-2ex}
\begin{lstlisting}
> x <- as.rowblock(x)
> x <- as.colblock(x)
> x <- redistribute(x, bldim=c(8, 8), ICTXT = 0)
\end{lstlisting}
  \end{block}
\end{frame}

\subsubsection{RandSVD}


\begin{frame}[fragile]
\fontsize{8pt}{10}\selectfont
\begin{block}{Randomized truncated SVD\footnotemark}
  \begin{minipage}{.56\textwidth}
    \begin{center}
      \includegraphics[height=.41\textheight]{../common/pics/randsvd/randSVDalg}
      \\
      \includegraphics[height=.26\textheight]{../common/pics/randsvd/randSVDalg4_4}
    \end{center}
  \end{minipage}
%   \hspace{.01cm}
  \begin{minipage}{0.43\textwidth}
\begin{lstlisting}[title=Serial
R,basicstyle=\tiny,backgroundcolor=\color{grayish}
,numberstyle=\tiny\color{black},keywordstyle=\color{black},commentstyle=\color{
dkgreen},stringstyle=\color{black},escapeinside={(*@}{@*)}]
randSVD <- function(A, k, q=3)
  {
    ## Stage A
    Omega <- (*@ matrix(rnorm(n*2*k),@*)
      (*@ nrow=n, ncol=2*k) @*)
    Y <- A %*% Omega
    Q <- qr.Q(qr(Y))
    At <- t(A)
    for(i in 1:q)
      {
        Y <- At %*% Q
        Q <- qr.Q(qr(Y))
        Y <- A %*% Q
        Q <- qr.Q(qr(Y))
      }

    ## Stage B
    B <- t(Q) %*% A
    U <- La.svd(B)$u
    U <- Q %*% U
    U[, 1:k]
  }
\end{lstlisting} %balance$
\end{minipage}
{\fontsize{6pt}{10}\selectfont $^1$Halko, Martinsson,
  and Tropp. 2011. Finding structure with randomness: probabilistic
  algorithms  for constructing \\[-1ex] approximate matrix decompositions
  \emph{SIAM Review} \textbf{53} 217--288}
\end{block}
\end{frame}


\begin{frame}[fragile]
 \fontsize{8pt}{10}\selectfont
\begin{block}{Randomized truncated SVD}
  \hfill
  \begin{minipage}{0.430\textwidth}
\begin{lstlisting}[title=Serial
R,basicstyle=\tiny,backgroundcolor=\color{grayish}
,numberstyle=\tiny\color{black},keywordstyle=\color{black},commentstyle=\color{
dkgreen},stringstyle=\color{black},escapeinside={(*@}{@*)}]
randSVD <- function(A, k, q=3)
  {
    ## Stage A
    Omega <- (*@ \textcolor{blue}{matrix(rnorm(n*2*k),} @*)
      (*@ \textcolor{blue}{ nrow=n, ncol=2*k)} @*)
    Y <- A %*% Omega
    Q <- qr.Q(qr(Y))
    At <- t(A)
    for(i in 1:q)
      {
        Y <- At %*% Q
        Q <- qr.Q(qr(Y))
        Y <- A %*% Q
        Q <- qr.Q(qr(Y))
      }

    ## Stage B
    B <- t(Q) %*% A
    U <- La.svd(B)$u
    U <- Q %*% U
    U[, 1:k]
  }
\end{lstlisting} %balance$
  \end{minipage}
  \hfill
  \begin{minipage}{0.430\textwidth}
\begin{lstlisting}[title=Parallel pbdR,basicstyle=\tiny,backgroundcolor=\color{
grayish}, numberstyle=\tiny\color{black},keywordstyle=\color{black},
commentstyle=\color{dkgreen},stringstyle=\color{black},escapeinside={(*@}{@*)}]
randSVD <- function(A, k, q=3)
  {
    ## Stage A
    Omega <- (*@ \textcolor{blue}{ddmatrix("rnorm",} @*)
      (*@ \textcolor{blue}{nrow=n, ncol=2*k)} @*)
    Y <- A %*% Omega
    Q <- qr.Q(qr(Y))
    At <- t(A)
    for(i in 1:q)
      {
        Y <- At %*% Q
        Q <- qr.Q(qr(Y))
        Y <- A %*% Q
        Q <- qr.Q(qr(Y))
      }

    ## Stage B
    B <- t(Q) %*% A
    U <- La.svd(B)$u
    U <- Q %*% U
    U[, 1:k]
  }
\end{lstlisting}  % balancing $
  \end{minipage}
\hfill
\end{block}
\end{frame}

\begin{frame}
  \begin{block}{From journal to scalable code and scaling data in one day.}
    \begin{center}
      \includegraphics[width=.4\textwidth]{../common/pics/randsvd/randSVDspeedup}
      \hspace{1cm}
      \includegraphics[width=.4\textwidth]{../common/pics/randsvd/randSpeedupSVD}
    \end{center}
  \end{block}
\end{frame}


\subsubsection{Analytics and Machine Learning}

\begin{frame}{Analytics and Machine Learning}
  \begin{block}{\pbdR Infrastructure Enables Scalability of R Analytics}
  \begin{itemize}
  \item Already scalable in packages and examples
    \begin{itemize}
    \item Model-based clustering (including k-means)
    \item Regression model fitting
    \item Random forest ensemble learning
    \item Functional data analysis basis functions
    \item Principal components analysis
    \item Random projection truncated SVD
    \item $\ldots$ others in progress
    \end{itemize}
  \item Opportunities for scaling with \pbdR
    \begin{itemize}
    \item Many R packages become ``low hanging fruit'' for scaling with \pbdR
    \item Opportunities for new statistical inference in scaling
    \end{itemize}
  \end{itemize}
  \end{block}
\end{frame}