\section[Applications]{Example Applications}

\hidenum
\begin{frame}[noframenumbering]
\frametitle{Contents}
 \tableofcontents[currentsection,hideothersubsections,sectionstyle=show/hide]
\end{frame}
\shownum

\subsection{Principal Components Analysis}

\begin{frame}
\begin{block}{Empirical Orthogonal Functions in Climate Analysis}
\begin{itemize}
\item Computation and volume rendering of large-scale EOF coherent
  modes in rotating turbulent flow data, AGU Fall Meeting, December 2013
\end{itemize}
\end{block}
\end{frame}

\begin{frame}[fragile,scale,shrink]
  \begin{exampleblock}{Coherent Modes in Turbulent Flow}\pause
\begin{lstlisting}[title=Get and Redistribute the Data]
library(pbdDMAT, quiet = TRUE)
init.grid()

## load local data (file assumes 4 processors!)
g.dim <- c(64, 64, 1024)
my.dim <- g.dim / c(1, 1, comm.size())
save.file <- paste("xyz.RData", comm.rank(), sep="") # assumes 4 processors!
load(save.file)

## reshape 3d array into a matrix for PCA (EOF) computation
## first two dimensions become rows and third becomes columns
## local reshape dimensions
my.nrow <- prod(my.dim[1:2])
my.ncol <- my.dim[3]
ldim <- c(my.nrow, my.ncol)

## global reshape dimensions
g.nrow <- prod(g.dim[1:2])
g.ncol <- g.dim[3]
gdim <- c(g.nrow, g.ncol)

## now reshape local
X <- matrix(vx, nrow=my.nrow, ncol=my.ncol, byrow=FALSE)
Y <- matrix(vy, nrow=my.nrow, ncol=my.ncol, byrow=FALSE)
Z <- matrix(vz, nrow=my.nrow, ncol=my.ncol, byrow=FALSE)

## glue local pieces into a ddmatrix
X <- new("ddmatrix", Data=X, dim=gdim, ldim=ldim, bldim=ldim, ICTXT=1)
Y <- new("ddmatrix", Data=Y, dim=gdim, ldim=ldim, bldim=ldim, ICTXT=1)
Z <- new("ddmatrix", Data=Z, dim=gdim, ldim=ldim, bldim=ldim, ICTXT=1)

## transform to 2d block cyclic
X <- redistribute(X, bldim=c(8,8), ICTXT=0)
Y <- redistribute(Y, bldim=c(8,8), ICTXT=0)
Z <- redistribute(Z, bldim=c(8,8), ICTXT=0)
\end{lstlisting}
  \end{exampleblock}
\end{frame}

\begin{frame}[fragile,scale,shrink]
  \begin{exampleblock}{Coherent Modes in Turbulent Flow}\pause
\begin{lstlisting}[title=Compute PCA and do Scree Plot (e0\_pca.r)]
E <- sqrt(X^2 + Y^2 + Z^2) # energy from velocity
E.pca <- prcomp(x=E, retx=TRUE, scale=FALSE) 

# plot using one process
if(comm.rank() == 0)
  {
    ## scree plot for first 50 components
    variance <- E.pca$sdev^2  # note: all own sdev
    proportion <- variance[1:50]/sum(variance)
    cumulative <- cumsum(proportion)
    component <- 1:length(proportion)
    png("scree.png")
    plot(component, cumulative, ylim=c(0,1))
    points(component, proportion, type="h", col="blue")
    dev.off()
  }

finalize()
        
        

\end{lstlisting} %$
  \end{exampleblock}
\end{frame}



%%% I really don't think you want to do this...
% \setcounter{framenumber}{0}



\subsection{Parallel Plot Ensembles}

\begin{frame}
  \begin{block}{How can we plot in parallel?}\pause
  \begin{itemize}
    \item Several plots, one on each processor
    \item One plot by several processors
  \end{itemize}
  \end{block}
\end{frame}

\begin{frame}[fragile,scale,shrink]
  \begin{exampleblock}{Plots in parallel}\pause
\begin{lstlisting}[title=png.slice]
png.slice <- function(x, g.dim, lab="slice", title=lab, work.dir="", zero.center=TRUE, most.positive=TRUE)
{
  X <- array(as.vector(x), dim=g.dim)

  ## prepare zero centered topo.colors
  if(zero.center)
    {
       . . .
    }
  else
    zlim <- range(X)

  ## set most positive (for unique up to sign)
  if(most.positive)
    {
       . . .
    }
  
  ## make png file
  file <- paste(work.dir, lab, "-r", comm.rank(), ".png", sep="")
  png(file)
  image(x=1:g.dim[1], y=1:g.dim[2], z=X, col=topo.colors(40), useRaster=TRUE, asp=1, xlim=c(1, g.dim[1] + 1), ylim=c(1, g.dim[2] + 1), zlim=zlim)
  title(title)
  ret <- dev.off()
  invisible(ret)
}
\end{lstlisting}
  \end{exampleblock}
\end{frame}


\begin{frame}[fragile,scale,shrink]
  \begin{exampleblock}{Plots in parallel}\pause
\begin{lstlisting}[title=Getting the Data]
library(pbdDMAT, quiet = TRUE)
init.grid()

## set global and local dimensions
g.dim <- c(64, 64, 1024)
my.dim <- g.dim / c(1, 1, comm.size())

save.file <- paste("xyz.RData", comm.rank(), sep="")
load(save.file)    # gets vx vector

## reshape 3d array into a matrix
## first two dimensions become rows and third becomes columns

## local reshape dimensions
my.nrow <- prod(my.dim[1:2])
my.ncol <- my.dim[3]
ldim <- c(my.nrow, my.ncol)

## global reshape dimensions
g.nrow <- prod(g.dim[1:2])
g.ncol <- g.dim[3]
gdim <- c(g.nrow, g.ncol)

## now reshape local
X <- matrix(vx, nrow=my.nrow, ncol=my.ncol, byrow=FALSE)

## glue local pieces into a ddmatrix
X <- new("ddmatrix", Data=X, dim=gdim, ldim=ldim, bldim=ldim, ICTXT=1)

## transform to 2d block cyclic
X <- redistribute(X, bldim=c(8,8), ICTXT=0)

## Plot few columns in parallel
   . . .
finalize()
\end{lstlisting}
  \end{exampleblock}
\end{frame}

\begin{frame}[fragile,scale,shrink]
  \begin{exampleblock}{Plots in parallel}\pause
\begin{lstlisting}[title=Make comm.size() plots in parallel]
step <- 5
max.plots <- min(20, ncol(X) %/% step)
last.plot <- 1 - step
time <- comm.timer(
for(i in 1:max.plots)
    {
        now.plots <- last.plot + step*(1:comm.size())
        my.col <- gather.col(X[, now.plots])
        lab <- paste("col", lead0(now.plots[comm.rank() + 1]), sep="")
        png.slice(my.col, g.dim[1:2], lab)
        last.plot <- now.plots[length(now.plots)]
    }
)
\end{lstlisting}
  \end{exampleblock}
\end{frame}

\begin{frame}[fragile,scale,shrink]
  \begin{exampleblock}{Plots in parallel}\pause
\begin{lstlisting}[title=gather.col First Attempt (p0\_plot.r)]
gather.col <- function(x, num=min(ncol(x), comm.size()))
{
    ## gather complete columns of a global array to different ranks
    my.local <- as.vector(x[, comm.rank() + 1], proc.dest=comm.rank())
    my.local
}
\end{lstlisting}
  \end{exampleblock}
\end{frame}

\begin{frame}[fragile,scale,shrink]
  \begin{exampleblock}{Plots in parallel}\pause
\begin{lstlisting}[title=gather.col Second Attempt (p1\_plot.r)]
gather.col <- function(x, num=min(ncol(x), comm.size()))
{
  ## gather complete columns of a global array to different ranks
  my.local <- NULL
  for(i in 1:num)
    {
      ## serial collection of unique data to each rank
      local <- as.vector(x[, i])
      if(comm.rank() + 1 == i) my.local <- local
    }
  my.local
}
\end{lstlisting}
  \end{exampleblock}
\end{frame}

\begin{frame}[fragile,scale,shrink]
  \begin{exampleblock}{Plots in parallel}\pause
\begin{lstlisting}[title=gather.col The Right Way (p2\_plot.r)]
gather.col <- function(x, num=min(ncol(x), comm.size()))
{
  ## gather complete columns of a global array to different ranks
  x.num <- x[, 1:num]
  x.num <- as.colblock(x.num)

  ## ScaLAPACK fix (a future release will automate)
  if(ownany(x.num))
    ret <- as.vector(submatrix(x.num))
  else
    ret <- NULL
  ret
}
\end{lstlisting}
  \end{exampleblock}
\end{frame}


\begin{frame}
  \begin{block}{SimpleRedistributions}\pause
  \begin{itemize}
\item as.block(dx, square.bldim = TRUE)
\item as.rowblock(dx)
\item as.colblock(dx)
\item as.rowcyclic(dx, bldim = .BLDIM)
\item as.colcyclic(dx, bldim = .BLDIM)
\item as.blockcyclic(dx, bldim = .BLDIM)
  \end{itemize}
  \end{block}
  \begin{block}{BLACS context (Processor Grid)}\pause
  \begin{itemize}
\item init.grid(P,Q)
\item .ICTXT = 0 gives P x Q 
\item .ICTXT = 1 gives PQ x 1
\item .ICTXT = 2 gives 1 x PQ
  \end{itemize}
  \end{block}
\end{frame}

\begin{frame}
  \begin{block}{Exercise}\pause
  \begin{itemize}
  \item Experiment with scripts x0\-ictxt.r, x1\_ictxt.r, and x2\_ictxt.r
  \item Experiment with other redistributions
  \end{itemize}
  \end{block}
\end{frame}

