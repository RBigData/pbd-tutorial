%%%%%%%%%%%%%%%%%%%%%%%%%%%%%%%%%%%%%%%%
%%     Title and ToC
%%%%%%%%%%%%%%%%%%%%%%%%%%%%%%%%%%%%%%%%
% titlepage
\frame{
  \maketitle
}

\begin{frame}[noframenumbering]
\frametitle{Affiliations and Support}
{\small
The pbdR Core Team\\ \url{http://r-pbd.org}
\\[.4cm]
Wei-Chen Chen\footnote{\tiny{Computer Science and Mathematics Division, Oak Ridge National Laboratory, Oak Ridge, TN}}, 
George Ostrouchov$^{1,2}$, 
Pragneshkumar Patel\footnote{\tiny{Remote Data Analysis and Visualization Center, University of Tennessee, Knoxville, TN}}, 
Drew Schmidt$^1$
\\[.4cm]
Ostrouchov, Patel, and Schmidt were supported in part by the project
``NICS Remote Data Analysis and Visualization Center''
funded by the Office of Cyberinfrastructure of the
U.S. National Science Foundation
under Award No. ARRA-NSF-OCI-0906324 for NICS-RDAV center.\\[.4cm]
Chen and Ostrouchov were supported in part by the project
``Visual Data Exploration and Analysis of Ultra-large Climate Data''
funded by U.S. DOE Office of Science
under Contract No. DE-AC05-00OR22725.\\
}
\end{frame}

\begin{frame}
\frametitle{About This Presentation}
 \begin{block}{Downloads}
  This presentation and supplemental materials are available at:
  \begin{center}
  \url{http://r-pbd.org/tutorial}
  \end{center}
  Sample R scripts and pbs job scripts available on Nautilus from:\\
  \centering\code{/lustre/medusa/mschmid3/tutorial/scripts.tar.gz}
 \end{block}
\end{frame}


\begin{frame}
\frametitle{About This Presentation}
 \begin{block}{\emph{Speaking Serial R with a Parallel Accent}}
  The content of this presentation is based in part on the \pkg{pbdDEMO} 
vignette \emph{Speaking Serial R with a Parallel Accent}\\[.4cm]
  \url{http://goo.gl/HZkRt}\\[.4cm]
  It contains more examples, and sometimes added detail.
 \end{block}
\end{frame}


\begin{frame}
\frametitle{About This Presentation}
 \begin{block}{Installation Instructions}
  Installation instructions for setting up a pbdR environment are available:
  \begin{center}
  \url{http://r-pbd.org/install.html}
  \end{center}
  This includes instructions for installing R, MPI, and pbdR.
 \end{block}
\end{frame}



\begin{frame}[noframenumbering]
\frametitle{Contents}
\small
\tableofcontents[hideallsubsections]
\end{frame}

\setcounter{framenumber}{0}
