\section{Introduction to HPC and Its View from R}
\makesubcontentsslides

\subsection{An R and \protect\pbdR View of Parallel Hardware and Software}
\makesubcontentsslidessec

\begin{frame}{R Interfaces to Low-Level Native Tools}
\includegraphics[height=\textheight]
{../common/pics/hardware/ParallelHardware10.pdf}
\end{frame}

\begin{frame}{Big Data and Little Data}
\begin{minipage}{10cm}
  \includegraphics[height=0.9\textheight]
  {../common/pics/hardware/ParallelHardware22.pdf}\hfill
\end{minipage}
\begin{minipage}{5cm}\small
  \begin{block}{Analysis Workflow}\pause
    \begin{itemize}[<+-|alert@+>]
    \item Get Big Data
      \begin{itemize}
      \item Parallel data reader
      \item Parallel data generator
      \end{itemize}
    \item Write analysis script
    \item Graphics to display results
    \item Profile and optimize code
    \end{itemize}
  \end{block}
\end{minipage}
\end{frame}

\begin{frame}{R and \pbdR Interfaces to Libraries: Harness 30+ Years of
    HPC Research}
\includegraphics[height=1.05\textheight]
{../common/pics/hardware/ParallelHardware14.pdf}
\end{frame}

\begin{frame}{Hadoop: Will it merge with HPC in the future?}
\begin{minipage}{10cm}
  \includegraphics[height=0.9\textheight]
  {../common/pics/hardware/ParallelHardware23.pdf}
\end{minipage}
\begin{minipage}{5cm}\small
  \begin{block}{Three Hadoop components}\pause
    \begin{itemize}[<+-|alert@+>]
    \item File system: HDFS
    \item Resource manager: Yarn
    \item Algorithm limitation: Map Reduce, Manager+Workers
    \end{itemize}
  \end{block}
\end{minipage}
\end{frame}

\begin{frame}{R and \pbdR Interfaces to Libraries: ``Standing on the Shoulders of
  Giants''}
\includegraphics[height=1.05\textheight]
{../common/pics/hardware/ParallelHardware14.pdf}
\end{frame}

%\begin{frame}{Low level R Interfaces to Native Tools}
%\includegraphics[width=0.95\textheight]
%{../common/pics/hardware/ParallelHardware22.pdf}
%\includegraphics[width=0.95\textheight]
%{../common/pics/hardware/ParallelHardware23.pdf}
%\end{frame}



\subsection{Batch and Interactive}
\makesubcontentsslidessec

\begin{frame}
  \begin{block}{Data analysis is interactive!}
    \pause
    \begin{itemize}[<+-|alert@+>]
    \item Data reduction to knowledge
    \item Iterative process with same data
      \begin{itemize}
      \item Exploration, model construction
      \item Diagnostics of fit and quantification of uncertainty
      \item Interpretation
      \end{itemize}
    \item S (and R) interactive ``answer'' to batch data analysis
    \item Efficient use of expensive people
    \end{itemize}
  \end{block}
  \begin{block}{Big platform computing is batch!}
    \pause
    \begin{itemize}[<+-|alert@+>]
    \item Libraries built for batch computing
    \item Traditionally data generation by simulation science
    \item Efficient use of expensive platforms
    \end{itemize}
  \end{block}
\end{frame}

\begin{frame}
  \begin{block}{High-Level Language: Batch and Interactive Distinction Blurred.}
    \begin{itemize}
    \item A function is a ``batch'' script
    \item \R ``An interactive environment to use batch scripts''
    \end{itemize}
  \end{block}
  \begin{block}{Ideal solution: Interactive Client with a Batch
      Server} 
    \begin{itemize}
    \item Parallel visualization systems (VisIt and ParaView) are
      client-server (batch on server)
    \item Current \pbdR packages address server side (batch)
    \item \pbdR client under development
      \begin{itemize}
      \item SPMD interactive (pbdCS, alpha on GitHub)
      \item Bridge laptop to login node to resource manager to cluster
      \item Site configuration file
      \item Manage relationship of big data (server side) to little
        data (client side)
      \end{itemize}
    \end{itemize}
  \end{block}
\end{frame}


\subsection{Programming Models}
\makesubcontentsslidessec

\begin{frame}{Manager-Workers}
  \begin{block}{}
    \begin{itemize}
    \item A serial program (Manager) divides up work and/or data
    \item Workers run in parallel without interaction
    \item Manager collects/combines results from workers
    \item Divide-Recombine fits this model
    \end{itemize}
  \end{block}
\end{frame}

\begin{frame}{MapReduce}
  \begin{block}{}
    \begin{itemize}
    \item A concept born of a search engine
    \item Decouples certain coupled problems with an intermediate
      communication - shuffle
    \item User writes two serial codes: Map and Reduce
    \end{itemize}
  \end{block}
\end{frame}

\begin{frame}{MapReduce: a Parallel Search Engine Concept}
  \begin{block}{Search MANY documents \hfill Serve MANY users}
    \begin{center}\scriptsize
      \begin{equation*}
        \begin{array}{c@{\hspace{-2ex}}r@{\hspace{-2ex}}c}
          \begin{array}{c}
            \mbox{\scriptsize Web} \\
            \mbox{\scriptsize Pages} \\
            \mbox{\scriptsize (records)}
          \end{array} &
          \begin{array}{c}\tiny
            \\ \mbox{\tiny p0} \\ \mbox{\tiny p1} \\
            \mbox{\tiny p2} \\ \mbox{\tiny p3}
          \end{array} &
          \begin{array}{c}
            \mbox{\scriptsize Index Words (keys)} \\
            \left[
            \begin{array}{cccc}
              A_1 & A_2  & A_3 & A_4 \\
              \hline
              B_1 & B_2  & B_3 & B_4 \\
              \hline
              C_1 & C_2  & C_3 & C_4 \\
              \hline
              D_1 & D_2  & D_3 & D_4
            \end{array}
            \right]
          \end{array}
        \end{array}
        \hbox{\hspace{-2ex}}
        \begin{array}{c}
          \hbox{Shuffle} \\
          \longrightarrow \\
          \mbox{\code{MPI\_Alltoallv}}
        \end{array}
        \hbox{\hspace{-2ex}}
        \begin{array}{c@{\hspace{-2ex}}r@{\hspace{-2ex}}c}
          \begin{array}{c}
            \mbox{\scriptsize Index} \\
            \mbox{\scriptsize Words} \\
            \mbox{\scriptsize (keys)}
          \end{array} &
          \begin{array}{c}
            \\  \mbox{\tiny p0} \\ \mbox{\tiny p1} \\
            \mbox{\tiny p2} \\ \mbox{\tiny p3}
          \end{array} &
          \begin{array}{c}
            \mbox{\scriptsize Web Pages (records)} \\
            \left[
            \begin{array}{cccc}
              A_1 & B_1  & C_1 & D_1 \\
              \hline
              A_2 & B_2  & C_2 & D_2 \\
              \hline
              A_3 & B_3  & C_3 & D_3 \\
              \hline
              A_4 & B_4  & C_4 & D_4
            \end{array}
            \right]
          \end{array}
        \end{array}
      \end{equation*}
    \end{center}
    \vspace{2em}
    \begin{center}
      Matrix transpose in another language?
    \end{center}
  \end{block}
\end{frame}

\begin{frame}
  \begin{block}{Can use different sets of processors}
    \begin{center}
      \begin{equation*}\scriptsize
        \begin{array}{c@{\hspace{-2ex}}r@{\hspace{-2ex}}c}
          \begin{array}{c}
            \mbox{\scriptsize Web} \\
            \mbox{\scriptsize Pages} \\
            \mbox{\scriptsize (records)}
          \end{array} &
          \begin{array}{c}\tiny
            \\ \mbox{\tiny p0} \\ \mbox{\tiny p1} \\
            \mbox{\tiny p2} \\ \mbox{\tiny p3}
          \end{array} &
          \begin{array}{c}
            \mbox{\scriptsize Index Words (keys)} \\
            \left[
            \begin{array}{cccc}
              \\
              \hline
              B_1 & B_2  & B_3 & B_4 \\
              \hline
              \\
              \hline
              \\
            \end{array}
            \right]
          \end{array}
        \end{array}
        \hbox{\hspace{-2ex}}
        \begin{array}{c}
          \hbox{Streaming} \\
          \hbox{Shuffle} \\
          \longrightarrow \\
          \mbox{\code{MPI\_Scatter}}
        \end{array}
        \hbox{\hspace{-2ex}}
        \begin{array}{c@{\hspace{-2ex}}r@{\hspace{-2ex}}c}
          \begin{array}{c}
            \mbox{\scriptsize Index} \\
            \mbox{\scriptsize Words} \\
            \mbox{\scriptsize (keys)}
          \end{array} &
          \begin{array}{c}
            \\  \mbox{\tiny p4} \\ \mbox{\tiny p5} \\
            \mbox{\tiny p6} \\ \mbox{\tiny p7}
          \end{array} &
          \begin{array}{c}
            \mbox{\scriptsize Web Pages (records)} \\
            \left[
            \begin{array}{cccc}
              \quad  & B_1  & \quad & \quad \\
              \hline
              \quad  & B_2  & \quad  &  \quad \\
              \hline
              \quad  & B_3  & \quad  &  \quad \\
              \hline
              \quad  & B_4  & \quad  & \quad
            \end{array}
            \right]
          \end{array}
        \end{array}
      \end{equation*}
    \end{center}
  \end{block}
\end{frame}

\begin{frame}{MPI and MapReduce}
  \begin{block}{Both Concepts are about Communication}
    \begin{itemize}
    \item One makes communication explicit, gives choices
    \item The other hides communication, gives one choice (shuffle)
    \end{itemize}
  \end{block}
\end{frame}

\begin{frame}{SPMD: Single Program Multiple Data}
  \begin{block}{}
    \begin{itemize}
    \item The prevalent way of distributed programming
    \item Can handle tightly coupled parallel computations
    \item It is designed for batch computing
    \item There is usually no manager - rather, all cooperate
    \item Prime driver behind MPI specification
    \end{itemize}
  \end{block}
\end{frame}

\begin{frame}{Early SPMD Work in Statistics: Crossproduct (Row-Block)}
  \includegraphics[width=\textwidth]
  {../common/pics/comm/Crossprod1987.png} \\
  \begin{block}{Hypercube: Individual send() and recv() over each dimension}
    {\scriptsize Ostrouchov (1987). Parallel Computing on a
      Hypercube: An overview of the architecture and some
      applications. {\em Proceedings of the 19th Symposium on the
        Interface of Computer Science and Statistics}, p.27-32.}
  \end{block}
\end{frame}

\begin{frame}{Simplified with MPI (and further with pbdMPI)}
  \includegraphics[trim=0cm 6cm 0cm 4cm,clip=true,width=\textwidth]
  {../common/pics/comm/ParallelHardware30.pdf}
  \vspace{-1ex}
  \begin{block}{Architecture-specific vendor optimizations}
    \begin{itemize}
    \item \small Cray MPT
    \item \small SGI MPT
    \end{itemize}
  \end{block}
\end{frame}

\begin{frame}{Data-flow: Parallel Runtime Scheduling and Execution
    Controller (PaRSEC)}
  \vspace{-.1cm}
  \hspace{2cm}\includegraphics[trim=0cm 0cm 0cm
  1cm,clip=true,width=7.5cm]{../common/pics/comm/PaRSEC1.png}
 \\[-3.4cm]
  \includegraphics[width=4cm]{../common/pics/comm/PaRSEC2.png}
  \hspace{5cm}{\tiny Graphic from icl.cs.utk.edu}
  \begin{block}
    {\tiny Bosilca, G., Bouteiller, A., Danalis, A., Faverge,
      M., Herault, T., Dongarra, J. "PaRSEC: Exploiting Heterogeneity
      to Enhance Scalability," IEEE Computing in Science and
      Engineering, Vol. 15, No. 6, 36-45, November, 2013.}
    \begin{itemize}\small
    \item Master data-flow controller runs distributed on all cores.
    \item Dynamic generation of current level in flow graph
    \item Effectively removes collective synchronizations
    \end{itemize}
  \end{block}
\end{frame}
