%%%%%%%%%%%%%%%%%%%%%%%%%%%%%%%%%%%%%%%%
%%     Title and ToC
%%%%%%%%%%%%%%%%%%%%%%%%%%%%%%%%%%%%%%%%
% titlepage
\frame{
  \maketitle
}

\setcounter{footnote}{0}

\begin{frame}[noframenumbering]
\frametitle{The \pbdR\ Core Team}
\begin{minipage}{1\textwidth}
  \vspace{-.6cm}
\begin{minipage}{3.6cm}
\ \\[.8cm]
Wei-Chen Chen$^1$ \\
George Ostrouchov$^{2}$ \\
Pragneshkumar Patel$^3$ \\
Drew Schmidt$^3$\\[2ex]
\end{minipage}
\begin{minipage}{7cm}
  \ \hfill \includegraphics[width=5.5cm]{../common/pics/newpbdr}
\end{minipage}
\end{minipage}

\vspace{-.4cm}
\begin{block}{Support}\tiny
  This work used resources of \textcolor{blue}{National Institute for
  Computational Sciences} at the University of Tennessee, Knoxville,
  which is supported by the Office of Cyberinfrastructure of the
  U.S. National Science Foundation under Award  No. ARRA-NSF-OCI-0906324 
  for NICS-RDAV center. 
  This work also used resources of the \textcolor{blue}{Oak Ridge
  Leadership Computing Facility} at the Oak Ridge National
  Laboratory, which is supported by the Office of Science of the
  U.S. Department of Energy under Contract No. DE-AC05-00OR22725.\\[.2cm]
  
  $^1$Department of Ecology and Evolutionary Biology\\
  University of Tennessee, Knoxville TN, USA\\[.2cm]

  $^2$Computer Science and Mathematics Division\\
  Oak Ridge National Laboratory, Oak Ridge TN, USA\\[.2cm]
  
  $^3$ National Institute for Computational Sciences\\
  University of Tennessee, Knoxville TN, USA
\end{block}
\end{frame}



% \begin{frame}
% \frametitle{About This Presentation}
%  \begin{block}{Downloads}
%   This presentation and supplemental materials are available at:
%   \begin{center}
%   \url{http://r-pbd.org/tutorial}
%   \end{center}
%  \end{block}
% \end{frame}



% \begin{frame}
% \frametitle{About This Presentation}
%   \begin{block}{Tutorial Evaluations}
%     \begin{center}
%       \url{http://bit.ly/sc13-tut-mf08}
%     \end{center}
%   \end{block}
% \end{frame}



\begin{frame}
\frametitle{About This Presentation}
 \begin{block}{\emph{Speaking Serial R with a Parallel Accent}}
  The content of this presentation is based in part on the \pkg{pbdDEMO} 
vignette \emph{Speaking Serial R with a Parallel Accent}\\[.4cm]
  \url{http://goo.gl/HZkRt}\\[.4cm]
  It contains more examples, and sometimes added detail.
 \end{block}
\end{frame}



\begin{frame}
\frametitle{About This Presentation}
 \begin{block}{Installation Instructions}
  Installation instructions for setting up a pbdR environment are available:
  \begin{center}
  \url{http://r-pbd.org/install.html}
  \end{center}
  This includes instructions for installing R, MPI, and pbdR.
 \end{block}
\end{frame}



% \begin{frame}%[allowframebreaks=0.8]
% \frametitle{About This Presentation}
%  \begin{block}{Conventions}
%   \begin{itemize}
%     \item We use ``{\Huge$ .$}'' as a decimal mark, not ``{\Huge$,$}''.  E.g., 
% 	``one thousand and one half'' is written ``$1,000.5$'', not ``$1.000,5$''.
%     \item We will use special suffixes to denote distributed objects (ones not 
% 	stored entirely on a single processor).\\
%     \code{.spmd} denotes a distributed object, while\\
%     \code{.dmat} denotes a distributed object which is of class 
% 	\code{ddmatrix}\\
%     No suffix means the object is global (common to all processors)\\[.2cm]
%     Neither of these suffices carries semantic meaning.
%     \end{itemize}
%  \end{block}
% \end{frame}



% \begin{frame}[fragile]
% \frametitle{About This Presentation}
%  \begin{block}{Conventions For Code Presentation}
% We will use two different forms of syntax highlighting.  One for displaying 
% results from an interactive R session:
% \begin{lstlisting}[backgroundcolor=\color{white},basicstyle=\ttfamily\color{
% dkgray}\scriptsize,keywordstyle=\color{black}, 
%   commentstyle=\color{orange},stringstyle=\color{mauve}]
% R> "interactive"
% [1] "interactive"
% \end{lstlisting}
% and one for presenting R scripts
% \begin{lstlisting}
% "not interactive"
% \end{lstlisting}
%  \end{block}
% \end{frame}



\begin{frame}[noframenumbering,shrink]
\frametitle{Contents}
\small
\tableofcontents[hideallsubsections]
\end{frame}

\setcounter{framenumber}{0}