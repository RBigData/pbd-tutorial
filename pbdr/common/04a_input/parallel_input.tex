\subsection{Parallel Data Input}
\makesubcontentsslidessec


\begin{frame}
  \begin{block}{Parallel reading brings new issues}\pause
    \begin{itemize}
    \item How to partition data across nodes?
    \item How to structure for scalable libraries?
    \item Read directly into form needed or restructure?
    \item CSV, SQL, HDFS, NetCDF4, HDF5, ADIOS, custom binary
    \item $\ldots$
    \item Lot of work needed to make it intuitive!
    \item Currently under development.
    \end{itemize}
  \end{block}
\end{frame}



\begin{frame}[fragile]
  \begin{exampleblock}{CSV Data}\pause
\begin{lstlisting}[title=Serial Code]
x <- read.csv("x.csv")

x
\end{lstlisting}

\begin{lstlisting}[title=Parallel Code]
library(pbdDEMO, quiet = TRUE)
init.grid()

dx <- read.csv.ddmatrix("x.csv", header=TRUE, sep=",", 
          nrows=10, ncols=10, num.rdrs=2, ICTXT=0)

dx

finalize()
\end{lstlisting}
  \end{exampleblock}
\end{frame}

\begin{frame}
  \begin{block}{Hadoop HDFS, RHadoop}\pause
    \begin{description}
      \item[ravro] - read and write files in avro format
      \item[plyrmr] - higher level plyr-like data processing for
        structured data, powered by rmr 
    \item[rmr] - functions providing Hadoop MapReduce functionality in
      R 
    \item[rhdfs] - functions providing file management of the HDFS
      from within R 
    \item[rhbase] - functions providing database management for the
      HBase distributed database from within R 
    \end{description}
  \end{block}
\end{frame}

\begin{frame}[fragile]
  \begin{exampleblock}{Binary Data: Vector}\pause
    \begin{lstlisting}
size <- 8 # bytes

my_ids <- get.jid(n)   # my index values from n

my_start <- (my_ids[1] - 1)*size    # my starting byte location
my_length <- length(my_ids)     # my number of bytes to read

con <- file("binary.vector.file", "rb")
seekval <- seek(con, where=my_start, rw="read")
x <- readBin(con, what="double", n=my_length, size=size)
    \end{lstlisting}
  \end{exampleblock}
\end{frame}

\begin{frame}[fragile]
  \begin{exampleblock}{Binary Data: Matrix}\pause \vspace{-0.8ex}
    \begin{lstlisting}[escapeinside={(*@}{@*)}]
size <- 8 # bytes

my_ids <- get.jid(ncol)
my_ncol <- length(my_ids)
my_start <- (my_ids[1] - 1)*(*@\textcolor{red}{nrow}@*)*size
my_length <- my_ncol*(*@\textcolor{red}{nrow}@*)

con <- file("binary.matrix.file", "rb")
seekval <- seek(con, where=my_start, rw="read")
x <- readBin(con, what="double", n=my_length, size=size)

gdim <- c(nrow, ncol)
ldim <- c(nrow, my_ncol)
bldim <- c(nrow, allreduce(my_ncol, op="max"))
X <- new("ddmatrix", Data=matrix(x, nrow, my_ncol), dim=gdim, ldim=ldim, bldim=bldim, ICTXT=1)    # glue together as column-block ddmatrix

X <- redistribute(X, bldim=c(2, 2), ICTXT=0)  # redistribute as block-cyclic
Xprc <- prcomp(X)   # proceed as with serial code
    \end{lstlisting}
  \end{exampleblock}
\end{frame}

\begin{frame}[fragile]
  \begin{exampleblock}{NetCDF4 Data}\pause \vspace{-0.8ex}
    \begin{lstlisting}
### parallel read after determining start and length
nc <- nc_open_par(file_name)

nc_var_par_access(nc, "variable_name")
new.X <- ncvar_get(nc, "variable_name", start, length)
nc_close(nc)
    \end{lstlisting}
  \end{exampleblock}
  \begin{exampleblock}{ADIOS Data (.bp files) \hfill pbdADIOS is under
    construction}\pause \vspace{-0.8ex}
    \begin{lstlisting}
### parallel read after determining start and length
file <- adios_open(file_name)

## Bounding box (start, length) access
## Staging capability with ADIOS configured simulation codes
## Streaming access

adios_close()
    \end{lstlisting}
  \end{exampleblock}
\end{frame}

