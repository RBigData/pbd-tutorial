\subsection{Parallel Data Input}

\begin{frame}
  \begin{block}{New Issues}\pause
    \begin{itemize}
    \item How to read in parallel?
    \item CSV, SQL, NetCDF4, HDF, ADIOS, custom binary
    \item How to partition data across nodes?
    \item How to structure for scalable libraries?
    \item Read directly into form needed or restructure?
    \item $\ldots$
    \item A lot of work needed here!
    \end{itemize}
  \end{block}
\end{frame}

\begin{frame}[fragile,scale,shrink]
  \begin{exampleblock}{CSV Data}\pause
\begin{lstlisting}[title=Serial Code]
d <- read.csv(``x.csv'')
\end{lstlisting}

\begin{lstlisting}[title=Parallel Code 0\_readcsv.r]
library(pbdDEMO, quiet = TRUE)
init.grid()
dx <- read.csv.ddmatrix("x.csv", header=TRUE,
          sep='','', nrows=10, ncols=10,
          num.rdrs=2, ICTXT=0)
comm.print(dx)
finalize()
\end{lstlisting}
  \end{exampleblock}
\end{frame}

\begin{frame}[fragile,scale,shrink]
  \begin{exampleblock}{NetCDF4 Data}\pause
    \begin{lstlisting}[title=Parallel Read]
### Must determine who will read what portion(s) and how to assemble them before reading

### parallel read after determining st and co
nc <- nc_open_par(file.name)

nc_var_par_access(nc, "TREFHT")
new.X.gbdc <- ncvar_get(nc, "TREFHT", start = st, count = co)
nc_close(nc)

finalize()
    \end{lstlisting}
  \end{exampleblock}
\end{frame}

\begin{frame}[fragile,scale,shrink]
  \begin{exampleblock}{Binary Data}\pause
    \begin{lstlisting}[title=Read subcube, reshape to ddmatrix, make
      block cyclic]
library(pbdDMAT, quiet = TRUE)
init.grid()

data.dim <- c(2048, 2048, 2048) # full data dimension
g.start <- c(1, 1, 513)         # global subcube corner
g.dim <- c(64, 64, 1024)        # global subcube extent

my.start <- g.start + c(0, 0, comm.rank()*my.dim[3])
my.dim <- g.dim / c(1, 1, comm.size())

size <- 4 # file is single precision floats

vx <- block3d.read(``filename'', data.dim, my.start, my.dim, size)

## local reshape dimensions
my.nrow <- prod(my.dim[1:2])
my.ncol <- my.dim[3]
ldim <- c(my.nrow, my.ncol)

## global reshape dimensions
g.nrow <- prod(g.dim[1:2])
g.ncol <- g.dim[3]
gdim <- c(g.nrow, g.ncol)

## reshape local
X <- matrix(vx, nrow=my.nrow, ncol=my.ncol, byrow=FALSE)

## glue local pieces into a ddmatrix
X <- new("ddmatrix", Data=X, dim=gdim, ldim=ldim, bldim=ldim, ICTXT=1)

## transform to 2d block cyclic
X <- redistribute(X, bldim=c(8,8), ICTXT=0)
    \end{lstlisting}
  \end{exampleblock}
\end{frame}

\begin{frame}[fragile,scale,shrink]
  \begin{exampleblock}{Binary Data}\pause
    \begin{lstlisting}[title=3d Block Binary Reader]
block3d.read <- function(file, data.dim, my.start, my.dim, size=4) {
  con.x <- file(file, "rb", blocking=TRUE)

  start <- sum((my.start - 1) * c(1, cumprod(data.dim)[-length(data.dim)]))

  x <- rep(NA, prod(my.dim))
        
  block <- 1:my.dim[1]
        
  for(j in 1:my.dim[3]) {
    sofar <- 0
    for(i in 1:my.dim[2]) {
      seek(con.x, where=start, rw="read", origin="start")
      x[block] <- readBin(con=con.x, what="numeric", n=my.dim[1], size=size)
      block <- block + my.dim[1]

      start <- start + data.dim[1]*size
      sofar <- sofar + data.dim[1]*size
    }
    start <- start - sofar + data.dim[1]*data.dim[2]*size
  }

  close(con.x)
  x
}
    \end{lstlisting}
  \end{exampleblock}
\end{frame}
