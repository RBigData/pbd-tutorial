
\subsection{pbdR Paradigms}

\begin{frame}
  \begin{block}{pbdR Paradigms}
  Programs that use pbdR utilize:
  \begin{itemize}[<+-|alert@+>]
   \item Batch execution
   \item Single Program/Multiple Data (SPMD) style
%    \item Object Oriented Programming (OOP)
   \\[.2cm]
   \end{itemize}
    And generally utilize:
   \begin{itemize}
   \item Data Parallelism
  \end{itemize}
  \end{block}
\end{frame}


\begin{frame}[fragile]
  \begin{block}{Batch Execution}\pause
    \begin{itemize}
      \item Non-interactive
      \item Use
\vspace{-.4cm}
\begin{lstlisting}[language=sh]
Rscript my_script.r
\end{lstlisting}
or\vspace{-.4cm}
\begin{lstlisting}[language=sh]
R CMD BATCH my_script.r
\end{lstlisting}
      \item In parallel:
\vspace{-.4cm}
\begin{lstlisting}[language=sh]
mpirun -np 2 Rscript my_par_script.r
\end{lstlisting}
    \end{itemize}
  \end{block}
\end{frame}


\begin{frame}
  \begin{block}{Single Program/Multiple Data (SPMD)}\pause
    \begin{itemize}
      \item SPMD is a programming \emph{paradigm}.
      \item Not to be confused with SIMD.
      \item SPMD utilizes MIMD architecture computers.
      \item Arguably the simplest extension of serial programming.
    \end{itemize}
  \end{block}
\end{frame}


\begin{frame}
  \begin{block}{Single Program/Multiple Data (SPMD)}\pause
    \begin{itemize}
      \item Difficult to describe, easy to do\dots
      \item Only one program is written, executed in batch on all processors.
      \item Different processors are autonomous; there is no manager.
      \item The dominant programming model for large machines.
    \end{itemize}
  \end{block}
\end{frame}
