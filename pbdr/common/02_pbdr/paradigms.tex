\subsection{Using pbdR}
\makesubcontentsslidessec

\begin{frame}
  \begin{block}{pbdR Paradigms}
  \pbdR\ programs are \R programs!\\[.4cm]
  Differences:
  \begin{itemize}[<+-|alert@+>]
   \item Batch execution (non-interactive).
   \item Parallel code utilizes Single Program/Multiple Data (SPMD) style
   \item Emphasizes data parallelism.
   \end{itemize}
  \end{block}
\end{frame}


\begin{frame}[fragile]
  \begin{block}{Batch Execution}\pause
    \begin{itemize}
      \item Running a serial \R program in batch:
\vspace{-.4cm}
\begin{lstlisting}[language=sh]
Rscript my_script.r
\end{lstlisting}
or\vspace{-.4cm}
\begin{lstlisting}[language=sh]
R CMD BATCH my_script.r
\end{lstlisting}
      \item Running a parallel (with MPI) \R program in batch:
\vspace{-.4cm}
\begin{lstlisting}[language=sh]
mpirun -np 2 Rscript my_par_script.r
\end{lstlisting}
    \end{itemize}
  \end{block}
\end{frame}


\begin{frame}
  \begin{block}{Single Program/Multiple Data (SPMD)}\pause
    \begin{itemize}
      \item SPMD is a programming \emph{paradigm}.
      \item Not to be confused with SIMD.
	\end{itemize}\centering
	\begin{minipage}[t]{.45\textwidth}
	\vspace{0pt}
	\begin{block}{Paradigms}
	  \textbf{Programming models}\\
	  OOP, Functional, SPMD, \dots
	\end{block}
	\end{minipage}
	\hspace{.2cm}
	\begin{minipage}[t]{.45\textwidth}
	\vspace{0pt}
	\begin{block}{SIMD}
	  \textbf{Hardware instructions}\\
	  MMX, SSE, \dots
	\end{block}
	\end{minipage}
     \\[.4cm]
  \end{block}
\end{frame}


\begin{frame}
  \begin{block}{Single Program/Multiple Data (SPMD)}\pause
   SPMD is arguably the simplest extension of serial programming.
    \begin{itemize}
      \item Only one program is written, executed in batch on all processors.
      \item Different processors are autonomous; there is no manager.
      \item Dominant programming model for large machines for 30 years.
    \end{itemize}
  \end{block}
\end{frame}
