\subsection{Managing a Communicator}

\begin{frame}
  \begin{block}{Message Passing Interface (MPI)}\pause
    \begin{itemize}
      \item \textit{MPI}: Standard for managing communications (data and instructions) between 
different nodes/computers.
      \item \textit{Implementations}:  OpenMPI, MPICH2, Cray MPT, \dots
      \item Enables parallelism (via communication) on distributed machines.
      \item \textit{Communicator}: manages communications between processors.
    \end{itemize}
  \end{block}
\end{frame}


\begin{frame}
  \begin{block}{MPI Operations (1 of 2)}\pause
    \begin{itemize}
      \item \textbf{Managing a Communicator}:  Create and destroy communicators.\\
      \code{init()} --- initialize communicator\\
      \code{finalize()} --- shut down communicator(s)
      \\[.4cm]
      %
      \item \textbf{Rank query}: determine the processor's position in the communicator.\\
      \code{comm.rank()} --- ``who am I?''\\
      \code{comm.size()} --- ``how many of us are there?''\\[.4cm]
      \item \textbf{Printing}:  Printing output from various ranks.\\
      \code{comm.print(x)}\\
      \code{comm.cat(x)}\\
      \textbf{WARNING}: only use these functions on \emph{results}, never on yet-to-be-computed 
things.\\
    \end{itemize}
  \end{block}
\end{frame}


\begin{frame}[fragile]
  \begin{exampleblock}{Quick Example 1}
  \centering
\begin{lstlisting}[title=Rank Query: 1\_rank.r]
library(pbdMPI, quietly = TRUE)
init()

my.rank <- comm.rank()
comm.print(my.rank, all.rank=TRUE)

finalize()
\end{lstlisting}
  \begin{columns}[t,onlytextwidth]
    \begin{column}{0.62\textwidth}
\begin{lstlisting}[backgroundcolor=\color{white},keywordstyle=\color{black},title=Execute this 
script via:]
mpirun -np 2 Rscript 1_rank.r
\end{lstlisting}    
    \end{column}
    \hfill
    \begin{column}{0.35\textwidth}
\begin{lstlisting}[title=Sample Output:]
COMM.RANK = 0
[1] 0
COMM.RANK = 1
[1] 1
\end{lstlisting}
    \end{column}
​  \end{columns}
  \end{exampleblock}
\end{frame}



\begin{frame}[fragile]
  \begin{exampleblock}{Quick Example 2}
\begin{lstlisting}[title=Hello World: 2\_hello.r]
library(pbdMPI, quietly=TRUE)
init()

comm.print("Hello, world")

comm.print("Hello again", all.rank=TRUE, quietly=TRUE)

finalize()
\end{lstlisting}
  \begin{columns}[t,onlytextwidth]
    \begin{column}{0.58\textwidth}
\begin{lstlisting}[backgroundcolor=\color{white},keywordstyle=\color{black},title=Execute this 
script via:]
mpirun -np 2 Rscript 2_hello.r
\end{lstlisting}    
    \end{column}
    \hfill
    \begin{column}{0.4\textwidth}
\begin{lstlisting}[title=Sample Output:]
COMM.RANK = 0
[1] "Hello, world"
[1] "Hello again"
[1] "Hello again"
\end{lstlisting}
    \end{column}
​  \end{columns}
  \end{exampleblock}
\end{frame}




