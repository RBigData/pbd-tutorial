\subsection{Other pbdMPI Tools}
\makesubcontentsslidessec


% \begin{frame}
%   \begin{block}{MPI Package Controls}
% The \code{.SPMD.CT} object allows for setting different package options with \pkg{pbdMPI}.  See the 
% entry \emph{SPMD Control} of the \pkg{pbdMPI} manual for information about the \code{.SPMD.CT} 
% object:
% \begin{center}
% { \small
% \url{http://cran.r-project.org/web/packages/pbdMPI/pbdMPI.pdf}
% }
% \end{center}
%   \end{block}
% \end{frame}

% \begin{frame}[fragile,shrink]
%   \begin{exampleblock}{Quick Example 5}
% \begin{lstlisting}[title=Barrier: 5\_barrier.r]
% library(pbdMPI, quiet = TRUE)
% init()
% 
% .SPMD.CT$msg.barrier <- TRUE
% .SPMD.CT$print.quiet <- TRUE
% 
% for (rank in 1:comm.size()-1){
%   if (comm.rank() == rank){
%     cat(paste("Hello", rank+1, "of", comm.size(), "\n"))
%   }
%   barrier()
% }
% 
% comm.cat("\n")
% 
% comm.cat(paste("Hello", comm.rank()+1, "of", comm.size(), "\n"), all.rank=TRUE)
% 
% finalize()
% \end{lstlisting}
%   \begin{columns}[t,onlytextwidth]
%     \begin{column}{0.58\textwidth}
% \begin{lstlisting}[backgroundcolor=\color{white},keywordstyle=\color{black},title=Execute this 
% script via:]
% mpirun -np 2 Rscript 5_barrier.r
% \end{lstlisting}
%     \end{column}
%     \hfill
%     \begin{column}{0.4\textwidth}
% \begin{lstlisting}[title=Sample Output:]
% Hello 1 of 2 
% Hello 2 of 2 
% \end{lstlisting}
%     \end{column}
% ​  \end{columns}
%   \end{exampleblock}
% \end{frame}


\begin{frame}
  \begin{block}{Random Seeds}\pause
  \pkg{pbdMPI} offers a simple interface for managing random seeds:
    \begin{itemize}
      \item \code{comm.set.seed(seed=1234, diff=FALSE)} --- All processors use 
the same seed \code{seed=1234}
      \item \code{comm.set.seed(seed=1234, diff=FALSE)} --- All processors use 
the same seed.
    \end{itemize}
  \end{block}
\end{frame}



% \begin{frame}[fragile,shrink]
%   \begin{exampleblock}{Quick Example 6}
% \begin{lstlisting}[title=Timing: 6\_timer.r]
% library(pbdMPI, quiet=TRUE)
% init()
% 
% comm.set.seed(diff=T)
% 
% test <- function(timed)
% {
%   ltime <- system.time(timed)[3]
%   
%   mintime <- allreduce(ltime, op='min')
%   maxtime <- allreduce(ltime, op='max')
%   meantime <- allreduce(ltime, op='sum')/comm.size()
% 
%   return(data.frame(min=mintime, mean=meantime, max=maxtime))
% }
% 
% times <- test(rnorm(1e6)) # ~7.6MiB of data
% comm.print(times)
% 
% finalize()
% \end{lstlisting}
%   \begin{columns}[t,onlytextwidth]
%     \begin{column}{0.58\textwidth}
% \begin{lstlisting}[backgroundcolor=\color{white},keywordstyle=\color{black},title=Execute this 
% script via:]
% mpirun -np 2 Rscript 6_timer.r
% \end{lstlisting}
% \end{column}
%     \hfill
%     \begin{column}{0.35\textwidth}
% \begin{lstlisting}[title=Sample Output:]
%    min  mean   max
% 1 0.17 0.173 0.176
% \end{lstlisting}
% \end{column}
% \end{columns}
%   \end{exampleblock}
% \end{frame}



\begin{frame}
  \begin{block}{Other Helper Tools}\pause
  \pkg{pbdMPI} Also contains useful tools for Manager/Worker and task parallelism codes:
    \begin{itemize}
      \item \textbf{Task Subsetting}: Distributing a list of jobs/tasks\\ 
      \code{get.jid(n)}
      %
      \item \textbf{*ply}:  Functions in the *ply family.\\
      \code{pbdApply(X, MARGIN, FUN, \dots)} --- analogue of \code{apply()}\\
      \code{pbdLapply(X, FUN, \dots)} --- analogue of \code{lapply()}\\
      \code{pbdSapply(X, FUN, \dots)} --- analogue of \code{sapply()}\\
    \end{itemize}
  \end{block}
\end{frame}



