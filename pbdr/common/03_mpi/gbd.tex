\section{Distributing Data}
\makesubcontentsslides


\subsection{A Way to Distribute Your Data}
\makesubcontentsslidessec


\begin{frame}
  \begin{block}{Distributing Data}
  \centering
\textbf{Problem:}  How to distribute the data
\begin{center}
    \begin{minipage}{0.47\textwidth}
     \begin{center}
      \begin{align*}
      x &= \left[
            \begin{array}{lll}
            x_{1,1} & x_{1,2} & x_{1,3} \\
            x_{2,1} & x_{2,2} & x_{2,3} \\
            x_{3,1} & x_{3,2} & x_{3,3} \\
            x_{4,1} & x_{4,2} & x_{4,3} \\
            x_{5,1} & x_{5,2} & x_{5,3} \\
            x_{6,1} & x_{6,2} & x_{6,3} \\
            x_{7,1} & x_{7,2} & x_{7,3} \\
            x_{8,1} & x_{8,2} & x_{8,3} \\
            x_{9,1} & x_{9,2} & x_{9,3} \\
            x_{10,1} & x_{10,2} & x_{10,3} \\
            \end{array}
      \right]_{10\times 3}
      \end{align*}
     \end{center}
    \end{minipage}
    \begin{minipage}{0.47\textwidth}
    \centering
      {\fontsize{12cm}{1cm}\selectfont ? }
    \end{minipage}
    \end{center}
  \end{block}
\end{frame}



\begin{frame}
\begin{block}{Distributing a Matrix Across 4 Processors: Block Distribution}
  \begin{columns}[t,onlytextwidth]
    \begin{column}{0.5\textwidth}
      \hfill Data \hfill\ 
      \begin{align*}
      x &= \left[
            \begin{array}{lll}
            \color{g11}x_{1,1} & \color{g11}x_{1,2} & \color{g11}x_{1,3} \\
            \color{g11}x_{2,1} & \color{g11}x_{2,2} & \color{g11}x_{2,3} \\
            \color{g11}x_{3,1} & \color{g11}x_{3,2} & \color{g11}x_{3,3} \\\hline
            \color{g12}x_{4,1} & \color{g12}x_{4,2} & \color{g12}x_{4,3} \\
            \color{g12}x_{5,1} & \color{g12}x_{5,2} & \color{g12}x_{5,3} \\
            \color{g12}x_{6,1} & \color{g12}x_{6,2} & \color{g12}x_{6,3} \\\hline
            \color{g13}x_{7,1} & \color{g13}x_{7,2} & \color{g13}x_{7,3} \\
            \color{g13}x_{8,1} & \color{g13}x_{8,2} & \color{g13}x_{8,3} \\
            \color{g13}x_{9,1} & \color{g13}x_{9,2} & \color{g13}x_{9,3} \\\hline
            \color{g21}x_{10,1} & \color{g21}x_{10,2} & \color{g21}x_{10,3} \\
            \end{array}
      \right]_{10\times 3}
      \end{align*}
    \end{column}
    \begin{column}{0.5\textwidth}
    \hfill Processors \hfill\ 
      \begin{align*}
      \begin{tabular}{l}
        \color{g11}0 \\
        \color{g12}1 \\
        \color{g13}2 \\
        \color{g21}3 
      \end{tabular}
      \end{align*}
    \end{column}
  \end{columns}
\end{block}
\end{frame}


\begin{frame}
\begin{block}{Distributing a Matrix Across 4 Processors: Local Load Balance}
  \begin{columns}[t,onlytextwidth]
    \begin{column}{0.5\textwidth}
      \hfill Data \hfill\ 
      \begin{align*}
      x &= \left[
            \begin{array}{lll}
            \color{g11}x_{1,1} & \color{g11}x_{1,2} & \color{g11}x_{1,3} \\
            \color{g11}x_{2,1} & \color{g11}x_{2,2} & \color{g11}x_{2,3} \\
            \color{g11}x_{3,1} & \color{g11}x_{3,2} & \color{g11}x_{3,3} \\\hline
            \color{g12}x_{4,1} & \color{g12}x_{4,2} & \color{g12}x_{4,3} \\
            \color{g12}x_{5,1} & \color{g12}x_{5,2} & \color{g12}x_{5,3} \\
            \color{g12}x_{6,1} & \color{g12}x_{6,2} & \color{g12}x_{6,3} \\\hline
            \color{g13}x_{7,1} & \color{g13}x_{7,2} & \color{g13}x_{7,3} \\
            \color{g13}x_{8,1} & \color{g13}x_{8,2} & \color{g13}x_{8,3} \\\hline
            \color{g21}x_{9,1} & \color{g21}x_{9,2} & \color{g21}x_{9,3} \\
            \color{g21}x_{10,1} & \color{g21}x_{10,2} & \color{g21}x_{10,3} \\
            \end{array}
      \right]_{10\times 3}
      \end{align*}
    \end{column}
    \begin{column}{0.5\textwidth}
    \hfill Processors \hfill\ 
      \begin{align*}
      \begin{tabular}{l}
        \color{g11}0 \\
        \color{g12}1 \\
        \color{g13}2 \\
        \color{g21}3 
      \end{tabular}
      \end{align*}
    \end{column}
  \end{columns}
\end{block}
\end{frame}



% 
% \begin{frame}[fragile]
%   \fontsize{8pt}{10}\selectfont
%   \begin{block}{The \code{GBD} Data Structure}\pause
%   Throughout the examples, we will make use of the Generalized Block Distribution, or \code{GBD} 
% distributed matrix structure.
%   \begin{columns}[c,onlytextwidth]
%     \begin{column}{0.64\textwidth}
%   \begin{enumerate}
%     \item \code{GBD} is \emph{distributed}.  No processor owns all the data.
%     \item \code{GBD} is \emph{non-overlapping}. Rows uniquely assigned to processors.
%     \item \code{GBD} is \emph{row-contiguous}.  If a processor owns one element of a row, it owns 
% the entire row.
%     \item \code{GBD} is globally \emph{row-major}, locally \emph{column-major}.
%     \item \code{GBD} is often \emph{locally balanced}, where each processor owns (almost) the same 
% amount of data.  But this is not required.
%     \end{enumerate}
%     \end{column}
%     \begin{column}{0.35\textwidth}
%       \begin{align*}
%       \left[
%             \begin{array}{lll}
%             \color{g11}x_{1,1} & \color{g11}x_{1,2} & \color{g11}x_{1,3} \\
%             \color{g11}x_{2,1} & \color{g11}x_{2,2} & \color{g11}x_{2,3} \\
%             \color{g11}x_{3,1} & \color{g11}x_{3,2} & \color{g11}x_{3,3} \\\hline
%             \color{g12}x_{4,1} & \color{g12}x_{4,2} & \color{g12}x_{4,3} \\
%             \color{g12}x_{5,1} & \color{g12}x_{5,2} & \color{g12}x_{5,3} \\
%             \color{g12}x_{6,1} & \color{g12}x_{6,2} & \color{g12}x_{6,3} \\\hline
%             \color{g13}x_{7,1} & \color{g13}x_{7,2} & \color{g13}x_{7,3} \\
%             \color{g13}x_{8,1} & \color{g13}x_{8,2} & \color{g13}x_{8,3} \\\hline
%             \color{g21}x_{9,1} & \color{g21}x_{9,2} & \color{g21}x_{9,3} \\\
%             \color{g21}x_{10,1} & \color{g21}x_{10,2} & \color{g21}x_{10,3} \\
%             \end{array}
%       \right]
%       \end{align*}
%     \end{column}
%   \end{columns}
%   \begin{enumerate}
%     \setcounter{enumi}{5}
%     \item The last row of the local storage of a processor is adjacent (by global row) to the first 
% row of the local storage of next processor (by communicator number) that owns data.
%     \item \code{GBD} is (relatively) easy to understand, but can lead to bottlenecks if you have 
% many more columns than rows.
%   \end{enumerate}
%   \end{block}
% \end{frame}




% \begin{frame}[fragile]
%   \begin{block}{Quick Comment for GBD}\pause
%     Local pieces of \code{GBD} distributed objects will be given the suffix \code{.gbd} to visually 
% help distinguish them from global objects.  This suffix carries no semantic meaning.
%   \end{block}
% \end{frame}


