\subsection{Profiling R Code}
\makesubcontentsslidessec


\begin{frame}
  \begin{block}{Timings}
  Getting simple timings as a basic measure of performance is easy, and 
valuable.
  \begin{itemize}
    \item \code{system.time()} --- timing blocks of code.
    \item \code{Rprof()} --- timing execution of \R functions.
    \item \code{Rprofmem()} --- reporting memory allocation in \R.
    \item \code{tracemem()} --- detect when a copy of an \R object is created.
    \item The \pkg{rbenchmark} package --- Benchmark comparisons.
  \end{itemize}
  \end{block}
\end{frame}


% 
% \begin{frame}
%   \begin{block}{Timings}
% To time a block of code, use \code{system.time()}:
% \begin{lstlisting}[language=Rinteractive]
% > system.time(sqrt(1:100000))
%    user  system elapsed 
%   0.001   0.000   0.001 
% \end{lstlisting}
%   \end{block}
% \end{frame}
