\subsection{Why Profile?}
\makesubcontentsslidessec


\begin{frame}
  \begin{block}{Performance and Accuracy}
  \begin{minipage}[t]{.25\textwidth}
    \vspace{0pt}
    \includegraphics[scale=.7]{../common/pics/linus}
  \end{minipage}
  \hfill
  \begin{minipage}[t]{.73\textwidth}
    \vspace{0pt}
    \begin{quote}
      Sometimes $\pi = 3.14$ is (a) infinitely faster than the ``correct'' 
answer and (b) the difference between the ``correct'' and the ``wrong'' answer 
is meaningless. \dots The thing is, some specious value of ``correctness'' is 
often irrelevant because it doesn't matter. While performance almost always 
matters. And I absolutely detest the fact that people so often dismiss 
performance concerns so readily. 
    \end{quote}
    \ \hfill --- Linus Torvalds, August 8, 2008
  \end{minipage}
  \end{block}
\end{frame}



\begin{frame}
  \begin{block}{Why Profile?}
  \begin{itemize}
    \item Because performance matters.
    \item Bad practices scale up!
    \item Your bottlenecks may surprise you.
    \item Because R is dumb.
    \item R users claim to be data people\dots so act like it!
  \end{itemize}
  \end{block}
\end{frame}


\begin{frame}[fragile]{Compilers often correct bad behavior\dots}
  \begin{center}
\begin{minipage}{.4\textwidth}
\vspace{0pt}
\begin{lstlisting}[title=A Really Dumb Loop,language=Rcpp]
int main(){
    int x, i;
    for (i=0; i<10; i++)
        x = 1;
    return 0;
}
\end{lstlisting}
\begin{lstlisting}[language=shl,title=clang -O3 example.c]
main:
        .cfi_startproc
# BB#0:
        xorl    %eax, %eax
        ret
\end{lstlisting}
\end{minipage}
\begin{minipage}{.57\textwidth}
\begin{lstlisting}[language=shl,title=clang example.c]
main:               
        .cfi_startproc
# BB#0:
        movl    $0, -4(%rsp)
        movl    $0, -12(%rsp)
.LBB0_1:            
        cmpl    $10, -12(%rsp)
        jge     .LBB0_4
# BB#2:             
        movl    $1, -8(%rsp)
# BB#3:             
        movl    -12(%rsp), %eax
        addl    $1, %eax
        movl    %eax, -12(%rsp)
        jmp     .LBB0_1
.LBB0_4:
        movl    $0, %eax
        ret
\end{lstlisting}
\end{minipage}

  \end{center}
\end{frame}
  
  
  
\begin{frame}[fragile]{R will not!}
    \begin{center}
\begin{minipage}[t]{.45\textwidth}
\begin{lstlisting}[title=Dumb Loop,escapeinside={(*@}{@*)}]
for (i in 1:n){
 (*@ \textcolor{red}{tA <- t(A)} @*)
  Y <- tA %*% Q
  Q <- qr.Q(qr(Y))
  Y <- A %*% Q
  Q <- qr.Q(qr(Y))
}  

Q
\end{lstlisting}
\end{minipage}
\begin{minipage}[t]{.45\textwidth}
\begin{lstlisting}[title=Better Loop,escapeinside={(*@}{@*)}]
(*@ \textcolor{red}{tA <- t(A)} @*)
 
for (i in 1:n){
  Y <- tA %*% Q
  Q <- qr.Q(qr(Y))
  Y <- A %*% Q
  Q <- qr.Q(qr(Y))
}  

Q
\end{lstlisting}
\end{minipage}
\end{center}
\end{frame}





\begin{frame}[fragile]{Example from a Real R Package}
    \begin{center}
\begin{minipage}[t]{.6\textwidth}
\vspace{-.9cm}
\begin{lstlisting}[title=Exerpt from Original function,escapeinside={(*@}{@*)}]
while(i<=N){
  for(j in 1:i){
    (*@ \textcolor{red}{d.k <- as.matrix(x)[l==j,l==j]}@*)
      ...
\end{lstlisting}
\vspace{-.3cm}
\begin{lstlisting}[title=Exerpt from Modified function,escapeinside={(*@}{@*)}]
(*@\textcolor{red}{x.mat <- as.matrix(x)}@*)
  
while(i<=N){
  for(j in 1:i){
    (*@ \textcolor{red}{d.k <- x.mat[l==j,l==j]}
@*)
      ...
\end{lstlisting}
\end{minipage}
\begin{minipage}[t]{.34\textwidth}
\vspace{1.1cm}
By changing just 1 line of code, performance of the main 
method improved by \textbf{over 350\%}!
\end{minipage}
\end{center}
\end{frame}


\begin{frame}
  \begin{block}{Some Thoughts}
    \begin{itemize}
      \item R is slow.
      \item Bad programmers are slower.
      \item R isn't very clever (compared to a compiler).
      \item The Bytecode compiler helps, but not nearly as much as a compiler.
    \end{itemize}
  \end{block}
\end{frame}

