\subsection{Parallel Plot Ensembles}
\makesubcontentsslidessec


\begin{frame}
  \begin{block}{How can we plot in parallel?}\pause
  \begin{itemize}
    \item Several plots, one or more on each processor (can do now)
    \item One plot by several processors (need to rewrite graphics)
  \end{itemize}
  \end{block}
\end{frame}

\begin{frame}[fragile,scale,shrink]
  \begin{exampleblock}{Plots in parallel}\pause
\begin{lstlisting}[title=png.slice]
png.slice <- function(x, g.dim, lab="slice", title=lab, work.dir="", zero.center=TRUE, most.positive=TRUE)
{
  X <- array(as.vector(x), dim=g.dim)

  ## prepare zero centered topo.colors
  if(zero.center)
    {
       . . .
    }
  else
    zlim <- range(X)

  ## set most positive (for unique up to sign)
  if(most.positive)
    {
       . . .
    }
  
  ## make png file
  file <- paste(work.dir, lab, "-r", comm.rank(), ".png", sep="")
  png(file)
  image(x=1:g.dim[1], y=1:g.dim[2], z=X, col=topo.colors(40), useRaster=TRUE, asp=1, xlim=c(1, g.dim[1] + 1), ylim=c(1, g.dim[2] + 1), zlim=zlim)
  title(title)
  ret <- dev.off()
  invisible(ret)
}
\end{lstlisting}
  \end{exampleblock}
\end{frame}


\begin{frame}[fragile,scale,shrink]
  \begin{exampleblock}{Plots in parallel}\pause
\begin{lstlisting}[title=Get and Redistribute the Data]
library(pbdDMAT, quiet = TRUE)
init.grid()

## set global and local dimensions
g.dim <- c(64, 64, 1024)
my.dim <- g.dim / c(1, 1, comm.size())

save.file <- paste("xyz.RData", comm.rank(), sep="")
load(save.file)    # gets vx vector

## reshape 3d array into a matrix
## first two dimensions become rows and third becomes columns

## local reshape dimensions
my.nrow <- prod(my.dim[1:2])
my.ncol <- my.dim[3]
ldim <- c(my.nrow, my.ncol)

## global reshape dimensions
g.nrow <- prod(g.dim[1:2])
g.ncol <- g.dim[3]
gdim <- c(g.nrow, g.ncol)

## now reshape local
X <- matrix(vx, nrow=my.nrow, ncol=my.ncol, byrow=FALSE)

## glue local pieces into a ddmatrix
X <- new("ddmatrix", Data=X, dim=gdim, ldim=ldim, bldim=ldim, ICTXT=1)

## transform to 2d block cyclic
X <- redistribute(X, bldim=c(8,8), ICTXT=0)

## Plot few columns in parallel
   . . .
finalize()
\end{lstlisting}
  \end{exampleblock}
\end{frame}

\begin{frame}[fragile,scale,shrink]
  \begin{exampleblock}{Plots in parallel}\pause
\begin{lstlisting}[title=Make plots in parallel comm.size() batches]
step <- 5
max.plots <- min(20, ncol(X) %/% step)
last.plot <- 1 - step
time <- comm.timer(
for(i in 1:max.plots)
    {
        now.plots <- last.plot + step*(1:comm.size())
        my.col <- gather.col(X[, now.plots])
        lab <- paste("col", lead0(now.plots[comm.rank() + 1]), sep="")
        png.slice(my.col, g.dim[1:2], lab)
        last.plot <- now.plots[length(now.plots)]
    }
)
\end{lstlisting}
  \end{exampleblock}
\end{frame}

% \begin{frame}[fragile,scale,shrink]
%   \begin{exampleblock}{Plots in parallel}\pause
% \begin{lstlisting}[title=gather.col First Attempt (1\_plot.r)]
% gather.col <- function(x, num=min(ncol(x), comm.size()))
% {
%     ## gather complete columns of a global array to different ranks
%     my.local <- as.vector(x[, comm.rank() + 1], proc.dest=comm.rank())
%     my.local
% }
% \end{lstlisting}
%   \end{exampleblock}
% \end{frame}

% \begin{frame}[fragile,scale,shrink]
%   \begin{exampleblock}{Plots in parallel}\pause
% \begin{lstlisting}[title=gather.col Second Attempt (2\_plot.r)]
% gather.col <- function(x, num=min(ncol(x), comm.size()))
% {
%   ## gather complete columns of a global array to different ranks
%   my.local <- NULL
%   for(i in 1:num)
%     {
%       ## serial collection of unique data to each rank
%       local <- as.vector(x[, i])
%       if(comm.rank() + 1 == i) my.local <- local
%     }
%   my.local
% }
% \end{lstlisting}
%   \end{exampleblock}
% \end{frame}

\begin{frame}[fragile,scale,shrink]
  \begin{exampleblock}{Plots in parallel}\pause
\begin{lstlisting}[title=gather.col  (3\_plot.r)]
gather.col <- function(x, num=min(ncol(x), comm.size()))
{
  ## gather complete columns of a global array to different ranks
  x.num <- x[, 1:num]
  x.num <- as.colblock(x.num)

  ## ScaLAPACK fix (a future release will automate)
  if(ownany(x.num))
    ret <- as.vector(submatrix(x.num))
  else
    ret <- NULL
  ret
}
\end{lstlisting}
  \end{exampleblock}
\end{frame}

\begin{frame}[fragile,scale,shrink]
  \begin{exampleblock}{Plots in parallel}\pause
\begin{lstlisting}[title=Now Plot the PCA Components (4\_plot.r)]
E <- sqrt(X^2 + Y^2 + Z^2)

E.pca <- prcomp(x=E, retx=TRUE, scale=FALSE)

## Use ranks 1 to n.pca to plot individual components in parallel
n.pca <- min(comm.size(), g.nrow)
my.col <- gather.col(E.pca$x, num=n.pca)

if(!is.null(my.col))
  {
    ## component plots on rank 1 to n.pca
    lab <- paste("pc", comm.rank(), sep="")
    title <- paste(lab, "sigma^2 =", variance[comm.rank() + 1])
    png.slice(my.col, g.dim[1:2], lab, title=title, work.dir=work.dir)
  }
\end{lstlisting} %$end
  \end{exampleblock}
\end{frame}

\begin{frame}
  \begin{block}{Exercise: scripts/pbdDMAT/dmat\_app}\pause
  \begin{itemize}
  \item Experiment with scripts 0\_pca.r, 1\_plot.r, 2\_plot.r,
    3\_plot.r, 4\_plot.r
  \end{itemize}
  \end{block}
\end{frame}

