
\subsection{pbdDMAT}

\begin{frame}[fragile]
  \begin{block}{The \code{ddmatrix} Data Structure}\pause
    The more complicated the processor grid, the more complicated the distribution.
  \end{block}
\end{frame}


\begin{frame}[shrink]
\begin{exampleblock}{\code{ddmatrix}: 2-dimensional Block-Cyclic with 6 Processors}
\begin{align*}
x &= \left[
      \begin{array}{ll|ll|ll|ll|l}
      \color{g11}x_{11} & \color{g11}x_{12} & \color{g12}x_{13} & \color{g12}x_{14} & 
\color{g13}x_{15} & \color{g13}x_{16} & \color{g11}x_{17} & \color{g11}x_{18} & \color{g12}x_{19}\\
      \color{g11}x_{21} & \color{g11}x_{22} & \color{g12}x_{23} & \color{g12}x_{24} & 
\color{g13}x_{25} & \color{g13}x_{26} & \color{g11}x_{27} & \color{g11}x_{28} & 
\color{g12}x_{29}\\\hline
      \color{g21}x_{31} & \color{g21}x_{32} & \color{g22}x_{33} & \color{g22}x_{34} & 
\color{g23}x_{35} & \color{g23}x_{36} & \color{g21}x_{37} & \color{g21}x_{38} & \color{g22}x_{39}\\
      \color{g21}x_{41} & \color{g21}x_{42} & \color{g22}x_{43} & \color{g22}x_{44} & 
\color{g23}x_{45} & \color{g23}x_{46} & \color{g21}x_{47} & \color{g21}x_{48} & 
\color{g22}x_{49}\\\hline
      \color{g11}x_{51} & \color{g11}x_{52} & \color{g12}x_{53} & \color{g12}x_{54} & 
\color{g13}x_{55} & \color{g13}x_{56} & \color{g11}x_{57} & \color{g11}x_{58} & \color{g12}x_{59}\\
      \color{g11}x_{61} & \color{g11}x_{62} & \color{g12}x_{63} & \color{g12}x_{64} & 
\color{g13}x_{65} & \color{g13}x_{66} & \color{g11}x_{67} & \color{g11}x_{68} & 
\color{g12}x_{69}\\\hline
      \color{g21}x_{71} & \color{g21}x_{72} & \color{g22}x_{73} & \color{g22}x_{74} & 
\color{g23}x_{75} & \color{g23}x_{76} & \color{g21}x_{77} & \color{g21}x_{78} & \color{g22}x_{79}\\
      \color{g21}x_{81} & \color{g21}x_{82} & \color{g22}x_{83} & \color{g22}x_{84} & 
\color{g23}x_{85} & \color{g23}x_{86} & \color{g21}x_{87} & \color{g21}x_{88} & 
\color{g22}x_{89}\\\hline
      \color{g11}x_{91} & \color{g11}x_{92} & \color{g12}x_{93} & \color{g12}x_{94} & 
\color{g13}x_{95} & \color{g13}x_{96} & \color{g11}x_{97} & \color{g11}x_{98} & \color{g12}x_{99}\\
      \end{array}
\right]_{9\times 9}
\end{align*}
\begin{align*}
\text{Processor grid = }\left|
      \begin{array}{lll}
      \color{g11}0 & \color{g12}1 & \color{g13}2\\
      \color{g21}3 & \color{g22}4 & \color{g23}5
      \end{array}
\right| &= 
\left|
      \begin{tabular}{lll}
      \color{g11}(0,0) & \color{g12}(0,1) & \color{g13}(0,2)\\
      \color{g21}(1,0) & \color{g22}(1,1) & \color{g23}(1,2)
      \end{tabular}
\right|
\end{align*}
\end{exampleblock}
\end{frame}


\begin{frame}[shrink]
\begin{exampleblock}{Understanding \code{ddmatrix}: Local View}
\begin{align*}
\left[
      \begin{array}{ll|ll}
      \color{g11}x_{11} & \color{g11}x_{12} & \color{g11}x_{17} & \color{g11}x_{18}\\
      \color{g11}x_{21} & \color{g11}x_{22} & \color{g11}x_{27} & \color{g11}x_{28}\\\hline
      \color{g11}x_{51} & \color{g11}x_{52} & \color{g11}x_{57} & \color{g11}x_{58}\\
      \color{g11}x_{61} & \color{g11}x_{62} & \color{g11}x_{67} & \color{g11}x_{68}\\\hline
      \color{g11}x_{91} & \color{g11}x_{92} & \color{g11}x_{97} & \color{g11}x_{98}\\
      \end{array}
\right]_{5\times 4}
\left[
      \begin{array}{ll|l}
      \color{g12}x_{13} & \color{g12}x_{14} & \color{g12}x_{19}\\
      \color{g12}x_{23} & \color{g12}x_{24} & \color{g12}x_{29}\\\hline
      \color{g12}x_{53} & \color{g12}x_{54} & \color{g12}x_{59}\\
      \color{g12}x_{63} & \color{g12}x_{64} & \color{g12}x_{69}\\\hline
      \color{g12}x_{93} & \color{g12}x_{94} & \color{g12}x_{99}\\
      \end{array}
\right]_{5\times 3}
\left[
      \begin{array}{ll}
      \color{g13}x_{15} & \color{g13}x_{16}\\
      \color{g13}x_{25} & \color{g13}x_{26}\\\hline
      \color{g13}x_{55} & \color{g13}x_{56}\\
      \color{g13}x_{65} & \color{g13}x_{66}\\\hline
      \color{g13}x_{95} & \color{g13}x_{96}\\
      \end{array}
\right]_{5\times 2}
\\
\left[
      \begin{array}{ll|ll}
      \color{g21}x_{31} & \color{g21}x_{32} & \color{g21}x_{37} & \color{g21}x_{38}\\
      \color{g21}x_{41} & \color{g21}x_{42} & \color{g21}x_{47} & \color{g21}x_{48}\\\hline
      \color{g21}x_{71} & \color{g21}x_{72} & \color{g21}x_{77} & \color{g21}x_{78}\\
      \color{g21}x_{81} & \color{g21}x_{82} & \color{g21}x_{87} & \color{g21}x_{88}\\
      \end{array}
\right]_{4\times 4}
\left[
      \begin{array}{ll|l}
      \color{g22}x_{33} & \color{g22}x_{34} & \color{g22}x_{39}\\
      \color{g22}x_{43} & \color{g22}x_{44} & \color{g22}x_{49}\\\hline
      \color{g22}x_{73} & \color{g22}x_{74} & \color{g22}x_{79}\\
      \color{g22}x_{83} & \color{g22}x_{84} & \color{g22}x_{89}\\
      \end{array}
\right]_{4\times 3}
\left[
      \begin{array}{ll}
      \color{g23}x_{35} & \color{g23}x_{36} \\
      \color{g23}x_{45} & \color{g23}x_{46} \\\hline
      \color{g23}x_{75} & \color{g23}x_{76} \\
      \color{g23}x_{85} & \color{g23}x_{86} \\
      \end{array}
\right]_{4\times 2}
\end{align*}
\begin{align*}
\text{Processor grid = }\left|
      \begin{array}{lll}
      \color{g11}0 & \color{g12}1 & \color{g13}2\\
      \color{g21}3 & \color{g22}4 & \color{g23}5
      \end{array}
\right| &= 
\left|
      \begin{tabular}{lll}
      \color{g11}(0,0) & \color{g12}(0,1) & \color{g13}(0,2)\\
      \color{g21}(1,0) & \color{g22}(1,1) & \color{g23}(1,2)
      \end{tabular}
\right|
\end{align*}
\end{exampleblock}
\end{frame}



\begin{frame}[fragile]
  \fontsize{8pt}{10}\selectfont
  \begin{block}{The \code{ddmatrix} Data Structure}\pause
  \begin{columns}[c,onlytextwidth]
    \begin{column}{0.59\textwidth}
      \begin{enumerate}
        \item \code{ddmatrix} is \emph{distributed}.  No one processor owns all of the matrix.
        \item \code{ddmatrix} is \emph{non-overlapping}. Any piece owned by one processor is owned by 
no 
other processors.\\\hrule
        \item \code{ddmatrix} can be row-contiguous or not, depending on the processor grid and 
blocking 
factor used.
        \item \code{ddmatrix} is locally column-major and globally, it depends\dots
      \end{enumerate}
    \end{column}
    \begin{column}{0.4\textwidth}
      \begin{align*}
\left[
      \begin{array}{ll|ll|l}
      \color{g11}x_{11} & \color{g11}x_{12} & \color{g12}x_{13} & \color{g12}x_{14} & 
\color{g13}x_{15} \\
      \color{g11}x_{21} & \color{g11}x_{22} & \color{g12}x_{23} & \color{g12}x_{24} & 
\color{g13}x_{25} \\\hline
      \color{g21}x_{31} & \color{g21}x_{32} & \color{g22}x_{33} & \color{g22}x_{34} & 
\color{g23}x_{35} \\
      \color{g21}x_{41} & \color{g21}x_{42} & \color{g22}x_{43} & \color{g22}x_{44} & 
\color{g23}x_{45} \\\hline
      \color{g11}x_{51} & \color{g11}x_{52} & \color{g12}x_{53} & \color{g12}x_{54} & 
\color{g13}x_{55} \\
      \color{g11}x_{61} & \color{g11}x_{62} & \color{g12}x_{63} & \color{g12}x_{64} & 
\color{g13}x_{65} \\\hline
      \color{g21}x_{71} & \color{g21}x_{72} & \color{g22}x_{73} & \color{g22}x_{74} & 
\color{g23}x_{75} \\
      \color{g21}x_{81} & \color{g21}x_{82} & \color{g22}x_{83} & \color{g22}x_{84} & 
\color{g23}x_{85} \\\hline
      \color{g11}x_{91} & \color{g11}x_{92} & \color{g12}x_{93} & \color{g12}x_{94} & 
\color{g13}x_{95} \\
      \end{array}
      \right]
      \end{align*}
    \end{column}
  \end{columns}
  \begin{enumerate}
    \setcounter{enumi}{5}
        \item \code{GBD} is a generalization of the one-dimensional block \code{ddmatrix} distribution. 
 
Otherwise there is no relation.
        \item \code{ddmatrix} is confusing, but very robust.
    \end{enumerate}
  \end{block}
\end{frame}












\begin{frame}
  \begin{block}{Pros and Cons of This Data Structure}\pause
  \begin{center}
    \begin{minipage}[t]{.45\textwidth}
      \begin{center}
      \begin{block}{Pros}
	\begin{itemize}
	\item Robust for matrix computations.
	\end{itemize}
      \end{block}
      \end{center}
    \end{minipage}\hspace{.5cm}
    \begin{minipage}[t]{.45\textwidth}
      \begin{center}
      \begin{block}{Cons}
	\begin{itemize}
	  \item Confusing layout.
	\end{itemize}
      \end{block}
      \end{center}
    \end{minipage}
    \\[.6cm]
    \emph{This is why we hide most of the distributed details.}
    \\[.4cm]
    The details are there if you want them (you don't want them).
  \end{center}
  \end{block}
\end{frame}


\begin{frame}[fragile]
  \begin{block}{Methods for class \code{ddmatrix}}\pause
    \pkg{pbdDMAT} has over 100 methods with \emph{identical} syntax to R:
    \begin{itemize}
     \item \code{\`{}[\`{}}, \code{rbind()}, \code{cbind()}, \dots
     \item \code{lm.fit()}, \code{prcomp()}, \code{cov()}, \dots
     \item \code{\`{}\%*\%\`{}}, \code{solve()}, \code{svd()}, \code{norm()}, \dots
     \item \code{median()}, \code{mean()}, \code{rowSums()}, \dots
    \end{itemize}
\begin{lstlisting}[title=Serial Code]
cov(x)
\end{lstlisting}
  
\begin{lstlisting}[title=Parallel Code]
cov(x)
\end{lstlisting}
  \end{block}
\end{frame}

\begin{frame}[fragile]
  \begin{block}{Comparing pbdMPI and pbdDMAT}\pause
\pkg{pbdMPI}:
  \begin{itemize}
     \item MPI $+$ sugar.
     \item \code{GBD} not the only structure \pkg{pbdMPI} can handle (just a useful convention).
     \end{itemize}
     
     \pkg{pbdDMAT}:
     \begin{itemize}
     \item More of a software package.
     \item The \code{ddmatrix} structure \emph{must} be used for \pkg{pbdDMAT}.
     \item If the data is not 2d block-cyclic compatible, \code{ddmatrix} will \emph{definitely} give the 
wrong 
answer.
    \end{itemize}
  \end{block}
\end{frame}

\begin{frame}[fragile]
  \begin{block}{Quick Comments for Using pbdDMAT}\pause
    \begin{enumerate}
      \item Start by loading the package:
\vspace{-.4cm}
\begin{lstlisting}
library(pbdDMAT, quiet = TRUE)
\end{lstlisting}
      \item Always initialize before starting and finalize when finished:
\vspace{-.4cm}
\begin{lstlisting}
init.grid()

# ...

finalize()
\end{lstlisting}
      \item   Distributed \code{ddmatrix} objects will be given the suffix \code{.dmat} to visually 
help 
distinguish them from global objects.  This suffix carries no semantic meaning.
    \end{enumerate}
  \end{block}
\end{frame}