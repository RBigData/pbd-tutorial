\begin{frame}[fragile]
\frametitle{DMAT Exercises}
\begin{enumerate}
 \item  Subsetting, selection, and filtering are basic matrix operations featured
in R. The following may look silly, but it is useful for data
processing.  Let \code{x.dmat <- ddmatrix(1:30, 10, 3)}.  Do the following:
\begin{itemize}
\item
  \code{y.dmat <- x.dmat[c(1, 5, 4, 3), ]} \\
  \code{y.dmat <- x.dmat[c(10:3, 5, 5), ]} \\
  \code{y.dmat <- x.dmat[1:5, 3:1]}
\\[.2cm]
\item
  \code{y.dmat <- x.dmat[x.dmat[, 2] > 13, ]} \\
  \code{y.dmat <- x.dmat[x.dmat[, 2] > x.dmat[, 3], ]} \\
  \code{y.dmat <- x.dmat[, x.dmat[2,] > x.dmat[3, ]]} \\
  \code{y.dmat <- x.dmat[c(1, 3, 5), x.dmat[, 2] > x.dmat[, 3]]}
\end{itemize}

\item The \code{prcomp()} method returns rotations for all components.  Computationally verify with several examples that these rotations are orthogonal, i.e., that their crossproduct is the identity matrix.
\end{enumerate}
\end{frame}