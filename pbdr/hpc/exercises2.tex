\begin{frame}[fragile,allowframebreaks=.9]
\frametitle{DMAT Exercises}
\begin{enumerate}
  \item  Subsetting, selection, and filtering are basic matrix operations featured
  in R. The following may look silly, but it is useful for data
  processing.  Let \code{x.dmat <- ddmatrix(1:30, 10, 3)}.  Do the following:
  \begin{itemize}
  \item
    \code{y.dmat <- x.dmat[c(1, 5, 4, 3), ]} \\
    \code{y.dmat <- x.dmat[c(10:3, 5, 5), ]} \\
    \code{y.dmat <- x.dmat[1:5, 3:1]}
  \\[.2cm]
  \item
    \code{y.dmat <- x.dmat[x.dmat[, 2] > 13, ]} \\
    \code{y.dmat <- x.dmat[x.dmat[, 2] > x.dmat[, 3], ]} \\
    \code{y.dmat <- x.dmat[, x.dmat[2,] > x.dmat[3, ]]} \\
    \code{y.dmat <- x.dmat[c(1, 3, 5), x.dmat[, 2] > x.dmat[, 3]]}
  \end{itemize}

  \item The method \code{crossprod()} is an optimized form of the crossproduct computation \code{t(x.dmat) \%*\% x.dmat}.  For this exercise, let \code{x.dmat <- ddmatrix(1:30, nrow=10, ncol=3)}.
  \begin{enumerate}
    \item Verify that these computations really do produce the same results.
    \item Time each operation.  Which is faster?
  \end{enumerate}

  \item The \code{prcomp()} method returns rotations for all components.  Computationally verify by example that these rotations are orthogonal, i.e., that their crossproduct is the identity matrix.

  \item 
\end{enumerate}
\end{frame}